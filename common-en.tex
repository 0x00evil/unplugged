%
% loading packages
%

\RequirePackage{ifpdf}
\RequirePackage{ifxetex}

\ifpdf
  \RequirePackage[pdftex,%
       bookmarksnumbered,%
              colorlinks,%
          linkcolor=blue,%
              hyperindex,%
        plainpages=false,%
       pdfstartview=FitH]{hyperref}
\else\ifxetex
  \RequirePackage[bookmarksnumbered,%
               colorlinks,%
           linkcolor=blue,%
               hyperindex,%
         plainpages=false,%
        pdfstartview=FitH]{hyperref}
\else
  \RequirePackage[dvipdfm,%
        bookmarksnumbered,%
               colorlinks,%
           linkcolor=blue,%
               hyperindex,%
         plainpages=false,%
        pdfstartview=FitH]{hyperref}
\fi\fi
%\usepackage{hyperref}

% other packages
%--------------------------------------------------------------------------
\usepackage{graphicx, color}
\usepackage{wrapfig}
\usepackage{subcaption}  %subfig is deprecated
\usepackage{multicol}
\usepackage[table]{xcolor} %for colored table
\usepackage{tikz}
\usetikzlibrary{patterns,matrix,positioning,shapes}

\usepackage{amsmath, amsthm, amssymb, amsbsy} % for math
\usepackage{textgreek, upgreek}
\usepackage{gensymb}
\usepackage{mathtools}
\usepackage{extarrows}
\usepackage{exercise} % for exercise
\usepackage{import} % for nested input

%
% for programming
%
\usepackage{verbatim}
\usepackage{fancyvrb}
\usepackage{listings}
\usepackage{algorithm} %to remove rules change to \usepackage[plain]{algorithm}
\usepackage[noend]{algpseudocode} %for pseudo code, include algorithmicsx automatically
\usepackage{appendix}
\usepackage{makeidx} % for index support
\usepackage{titlesec}
\usepackage{epigraph}

\usepackage{fontspec}
\usepackage{xunicode}
\usepackage{textcomp}
\usepackage{url}

% detect and select Chinese font
% ------------------------------
% fc-list :lang=zh    % list all Chinese fonts
% fc-list :mono       % list all mono fonts
% fc-cache            % refresh cache to load new installed fonts
\def\macmainfont{STSong}  % Under Mac OS X
\def\macmonofont{Monaco}
\def\winmainfont{SimSun} % Under Windows
\def\winmonofont{Consolas}
\def\linuxmainfont{WenQuanYi Micro Hei} % Under Linux
\def\linuxmainfont{Courier}

\suppressfontnotfounderror1 % Avoid setting exit code (error level) to break make process
\count255=\interactionmode
\batchmode

% main font
\let\cnmainft=\macmainfont
\font\thefont="\cnmainft"\space at 10pt
\ifx\thefont\nullfont
  \let\cnmainft=\winmainfont
  \font\thefont="\cnmainft"\space at 10pt
  \ifx\the\nullfont
    \let\cnmainft=\linuxmainfont
    \font\thefont="\cnmainft"\space at 10pt
    \ifx\the\nullfont
      \errorstopmode
      \errmessage{no suitable Chinese main font found}
    \fi
  \fi
\fi

% mono font
\let\monoft=\macmonofont
\font\thefont="\monoft"\space at 10pt
\ifx\thefont\nullfont
  \let\monoft=\winmonofont
  \font\thefont="\monoft"\space at 10pt
  \ifx\the\nullfont
    \let\monoft=\linuxmonofont
    \font\thefont="\monoft"\space at 10pt
    \ifx\the\nullfont
      \errorstopmode
      \errmessage{no suitable mono font found}
    \fi
  \fi
\fi

\interactionmode=\count255

\titleformat{\paragraph}
{\normalfont\normalsize\bfseries}{\theparagraph}{1em}{}
\titlespacing*{\paragraph}
{0pt}{3.25ex plus 1ex minus .2ex}{1.5ex plus .2ex}

% for literate Haskell code
\lstdefinestyle{Haskell}{
  flexiblecolumns=false,
  basewidth={0.5em,0.45em},
  morecomment=[l]--,
  literate={+}{{$+$}}1 {/}{{$/$}}1 {*}{{$*$}}1 {=}{{$=$}}1
           {>}{{$>$}}1 {<}{{$<$}}1 {\\}{{$\lambda$}}1
           {\\\\}{{\char`\\\char`\\}}1
           {->}{{$\rightarrow$}}2 {>=}{{$\geq$}}2 {<-}{{$\leftarrow$}}2
           {<=}{{$\leq$}}2 {=>}{{$\Rightarrow$}}2
           {\ .}{{$\circ$}}2 {\ .\ }{{$\circ$}}2
           {>>}{{>>}}2 {>>=}{{>>=}}2
           {|}{{$\mid$}}1
}

\lstloadlanguages{C, C++, Java, Lisp, Haskell, Python}

\lstset{
  basicstyle=\small\ttfamily,
  commentstyle=\rmfamily,
  texcl=true,
  showstringspaces = false,
  upquote=true,
  flexiblecolumns=false
}

\newcommand\doubleplus{+\kern-1.3ex+\kern0.8ex}

% ======================================================================


\def\BibTeX{{\rm B\kern-.05em{\sc i\kern-.025em b}\kern-.08em
    T\kern-.1667em\lower.7ex\hbox{E}\kern-.125emX}}

%
% mathematics
%
\newcommand{\be}{\begin{equation}}
\newcommand{\ee}{\end{equation}}
\newcommand{\bmat}[1]{\left( \begin{array}{#1} }
\newcommand{\emat}{\end{array} \right) }
\newcommand{\VEC}[1]{\mbox{\boldmath $#1$}}

% numbered equation array
\newcommand{\bea}{\begin{eqnarray}}
\newcommand{\eea}{\end{eqnarray}}

% equation array not numbered
\newcommand{\bean}{\begin{eqnarray*}}
\newcommand{\eean}{\end{eqnarray*}}

\newtheorem{theorem}{Theorem}[section]
\newtheorem{lemma}[theorem]{Lemma}
\newtheorem{proposition}[theorem]{Proposition}
\newtheorem{corollary}[theorem]{Corollary}
\newtheorem{definition}{Definition}[section]
\newtheorem{example}{Example}[section]

\makeatletter
\newcommand{\verbatimfont}[1]{\renewcommand{\verbatim@font}{\ttfamily#1}}
\makeatother

\setcounter{tocdepth}{4}
\setcounter{secnumdepth}{4}
