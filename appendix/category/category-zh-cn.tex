\ifx\wholebook\relax \else

\documentclass[UTF8]{article}

\usepackage[cn]{../../prelude}

\setcounter{page}{1}

\begin{document}

\title{范畴}

\author{刘新宇
\thanks{{\bfseries 刘新宇} \newline
  Email: liuxinyu95@gmail.com \newline}
  }

\maketitle
\fi

\markboth{范畴}{编程中的数学}

\chapter*{积与和的唯一性}
\phantomsection  % so hyperref creates bookmarks
\addcontentsline{toc}{chapter}{积和余积的唯一性}

积与和是唯一的么?下面的定理回答了这个问题:

\begin{lemma}
对于范畴$\pmb{C}$中的一对对象$A$和$B$,令下图中的对象和箭头

\begin{center}
\begin{tikzpicture}
  \matrix (m) [matrix of math nodes,
               row sep=2em, column sep=2em, minimum width=2em]{
    P &   &  &   & I \\
      & A &  & A &   \\
      & B &  & B &   \\
    Q &   &  &   & J \\};
  \path[-stealth]
    % left
    (m-1-1) edge node [above] {$p_A$} (m-2-2)
    (m-1-1) edge node [below] {$p_B$} (m-3-2)
    (m-4-1) edge node [above] {$q_A$} (m-2-2)
    (m-4-1) edge node [below] {$q_B$} (m-3-2)
    % right
    (m-2-4) edge node [above] {$i_A$} (m-1-5)
    (m-2-4) edge node [above] {$j_A$} (m-4-5)
    (m-3-4) edge node [below] {$i_B$} (m-1-5)
    (m-3-4) edge node [below] {$j_B$} (m-4-5);
\end{tikzpicture}
\end{center}

各自为一对

\begin{center}
积 \quad \quad \quad 余积
\end{center}

的楔形。则

\[
P, Q \quad \quad \quad I, J
\]

是同构的楔形,存在唯一的箭头:

\begin{center}
\begin{tikzpicture}
  \matrix (m) [matrix of math nodes,
               row sep=2em, column sep=3em, minimum width=2em]{
    P & & I \\
    Q & & J \\};
  \path[-stealth]
    % left
    (m-1-1.south west) edge node [left] {$f$} (m-2-1.north west)
    (m-2-1.north east) edge node [right] {$g$} (m-1-1.south east)
    % right
    (m-1-3.south west) edge node [left] {$f$} (m-2-3.north west)
    (m-2-3.north east) edge node [right] {$g$} (m-1-3.south east);
\end{tikzpicture}
\end{center}

使得:

\[
\begin{array}{lcl}
  \begin{cases}
    p_A = q_A \circ f & p_B = q_B \circ f \\
    q_A = p_A \circ g & q_B = p_B \circ g
  \end{cases}
  & \quad &
  \begin{cases}
    i_A = g \circ j_A & i_B = g \circ j_B \\
    j_A = f \circ i_A & j_B = f \circ i_B
  \end{cases}
\end{array}
\]

并且,$f$和$g$是互逆的同构箭头对。
\end{lemma}

\begin{proof}
我们只证明左侧积的部分,右侧和的证明与此类似。给定$A, B$,对象$Q$和两个箭头$q_A, q_B$构成的楔形是一个积。我们把对象$P$和一对箭头$p_A, p_B$看成是另一个楔形。根据积的定义,存在唯一的媒介箭头$f$满足:

\[
p_A = q_A \circ f \quad \text{和} \quad p_B = q_B \circ f
\]

交换$P$和$Q$(令$P$为积,$Q$为任意楔形),又可以得到:

\[
q_A = p_A \circ g \quad \text{和} \quad q_B = p_B \circ g
\]

这样就有:

\[
\begin{cases}
p_A \circ g \circ f = q_A \circ f = p_A \\
p_B \circ g \circ f = q_B \circ f = p_B
\end{cases}
\]

以及:

\[
\begin{cases}
q_A \circ f \circ g = p_A \circ g = q_A \\
q_B \circ f \circ g = p_B \circ g = q_B \\
\end{cases}
\]

因此:

\[
g \circ f = id_P \quad \quad \quad f \circ g = id_Q
\]

\end{proof}

这一结论说明,如果两个对象存在积(或者和)则它是唯一的。

\chapter*{集合的笛卡尔积和不相交并集构成积与和的证明}
\phantomsection  % so hyperref creates bookmarks
\addcontentsline{toc}{chapter}{集合的笛卡尔积和不相交并集构成积与和的证明}

\begin{proof}
我们可以用构造法证明。首先证明左侧积的部分。$A \times B$的笛卡尔积是来自两个集合的元素对的全部组合

\[
\{(a, b) | a \in A, b \in B\}
\]

我们定义两个特殊的箭头(函数)作为$p_A$和$p_B$

\[
\begin{cases}
fst\ (a, b) = a \\
snd\ (a, b) = b
\end{cases}
\]

考虑任意楔形

\begin{center}
\begin{tikzpicture}
  \matrix (m) [matrix of math nodes,
               row sep=2em, column sep=2em, minimum width=2em]{
      & A  \\
    X &    \\
      & B  \\};
  \path[-stealth]
    (m-2-1) edge node[above] {$p$} (m-1-2)
    (m-2-1) edge node[below] {$q$} (m-3-2);
\end{tikzpicture}
\end{center}

其中$p\ x = a, q\ x = b, x \in X$。举个具体的例子,$X$为$Int$, $A$为$Int$,$B$为$Bool$,而两个函数$p, q$的定义为:

\[
\begin{cases}
p(x) = -x \\
q(x) = even(x)
\end{cases}
\]

也就是说,$p$对一个整数取相反数,$q$判断其是否是偶数。我们定义函数$X \arrowto{m} A \times B$如下:

\[
m(x) = (a, b)
\]

具体到上面的例子,有:

\[
m(x) = (-x, even(x))
\]

这样,下面图中的箭头就可交换了:

\begin{center}
\begin{tikzpicture}
  \matrix (m) [matrix of math nodes,
               row sep=2em, column sep=2em, minimum width=2em]{
      & A \\
    X & A \times B \\
      & B \\};
  \path[-stealth]
    % left
    (m-2-1) edge node [above] {$p$} (m-1-2)
    (m-2-1) edge node [below] {$q$} (m-3-2)
    (m-2-1) edge node [above] {$m$}   (m-2-2)
    (m-2-2) edge node [right] {$fst$} (m-1-2)
    (m-2-2) edge node [right] {$snd$} (m-3-2);
\end{tikzpicture}
\end{center}

不难验证:

\[
\begin{array}{rcll}
(fst \circ m)(x) & = & fst\ m(x) & \text{函数组合} \\
                 & = & fst\ (a, b) & \text{$m$的定义} \\
                 & = & a & \text{$fst$的定义} \\
                 & = & p(x) & \text{反向用$p$的定义}
\end{array}
\]

并且:

\[
\begin{array}{rcll}
(snd \circ m)(x) & = & snd\ m(x) & \text{函数组合} \\
                 & = & snd\ (a, b) & \text{$m$的定义} \\
                 & = & b & \text{$snd$的定义} \\
                 & = & q(x) & \text{反向用$q$的定义}
\end{array}
\]

具体到上面的例子,有:

\[
\begin{cases}
(fst \circ m)(x) = fst\ (-x, even(x)) = -x = p(x) \\
(snd \circ m)(x) = snd\ (-x, even(x)) = even(x) = q(x)
\end{cases}
\]

我们还需要证明$m$是唯一的。假设存在另一个函数$x \arrowto{h} A \times B$,也使得范畴图中的箭头可交换。即:

\[
fst \circ h = p \quad \text{且} \quad snd \circ h = q
\]

我们有:

\[
\begin{array}{rcll}
(a, b) & = & (p(x),\ q(x)) & \text{$p, q$的定义} \\
       & = & ((fst \circ h)(x),\ (snd \circ h)(x)) & \text{可交换} \\
       & = & (fst\ h(x),\ snd\ h(x)) & \text{函数组合} \\
       & = & (fst\ (a, b),\ snd\ (a, b)) & \text{反向用$fst, snd$的定义}
\end{array}
\]

所以$h(x) = (a, b) = m(x)$。这样就证明了$m$的唯一性。

接下来,我们证明右侧和的部分。不相交的并集$A+B$中的元素分为两类。一类是来自$A$中的元素$(a, 0)$,另一类是来自$B$中的元素$(b, 1)$。为此,我们定义两个特殊的箭头(函数)作为$i_A, i_B$:

\[
\begin{cases}
left(a) = (a, 0) \\
right(b) = (b, 1)
\end{cases}
\]

考虑任意集合$X$构成的楔形:

\begin{center}
\begin{tikzpicture}
  \matrix (m) [matrix of math nodes,
               row sep=2em, column sep=2em, minimum width=2em]{
    A & \\
      & X \\
    B & \\};
  \path[-stealth]
    (m-1-1) edge node [above] {$p$} (m-2-2)
    (m-3-1) edge node [below] {$q$} (m-2-2);
\end{tikzpicture}
\end{center}

我们定义箭头$A + B \arrowto{m} X$如下:

\[
\begin{cases}
m\ (a, 0) = p(a) \\
m\ (b, 1) = q(b)
\end{cases}
\]

这样,下图中的箭头就可交换了。

\begin{center}
\begin{tikzpicture}
  \matrix (m) [matrix of math nodes,
               row sep=2em, column sep=2em, minimum width=2em]{
    A & \\
    A + B & X \\
    B & \\};
  \path[-stealth]
    (m-1-1) edge node [above] {$p$} (m-2-2)
    (m-3-1) edge node [below] {$q$} (m-2-2)
    (m-2-1) edge node [above] {$m$}   (m-2-2)
    (m-1-1) edge node [left] {$left$} (m-2-1)
    (m-3-1) edge node [left] {$right$} (m-2-1);
\end{tikzpicture}
\end{center}

接下来要证明$m$的唯一性,也就是说$m$是唯一使上图可交换的箭头。假设存在另一个箭头$A + B \arrowto{h} X$也使得上图可交换。即满足:

\[
h \circ left = p \quad \text{且} \quad h \circ right = q
\]

我们任取$a \in A, b \in B$,有:

\[
\begin{cases}
h\ (a, 0) = h(left(a)) = (h \circ left)(a) = p(a) = m\ (a, 0) \\
h\ (b, 1) = h(right(a)) = (h \circ right)(b) = q(b) = m\ (b, 1)
\end{cases}
\]

于是$h = m$,这就证明了$m$的唯一性。
\end{proof}

\ifx\wholebook\relax \else

\expandafter\enddocument
%\end{document}

\fi
