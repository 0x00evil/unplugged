\ifx\wholebook\relax \else

\documentclass[UTF8]{article}

\usepackage[cn]{../../prelude}

\setcounter{page}{1}

\begin{document}

\fi

\markboth{参考文献}{编程中的数学}
\phantomsection  % so hyperref creates bookmarks
\addcontentsline{toc}{chapter}{参考文献}

\begin{thebibliography}{99}

% 自然数
% =====================================

\bibitem{wiki-number}
Wikipedia. ``古代计数系统的历史''. \url{https://en.wikipedia.org/wiki/History_of_ancient_numeral_systems}

\bibitem{trip-to-number-kindom}
[美]卡尔文$\cdot$C$\cdot$克劳森. ``数学旅行家:漫游数王国''. 袁向东、袁钧译,上海教育出版社。ISBN: 7-5320-7883-3/G $cdot$ 7972

\bibitem{wiki-babylonian-num}
Wikipedia. ``古巴比伦数字''. \url{https://en.wikipedia.org/wiki/Babylonian_numerals}

\bibitem{M-Kline-2007}
[美] M$\cdot$克莱因 著 李宏魁 译 ``数学:确定性的丧失'' 湖南科学技术出版社,2007年4月 ISBN: 978-7-5357-1857-0
% Morris Kline ``Mathematics: The Loss of Certainty''. Oxford University Press, 1980.

\bibitem{GEB}
[美]候世达 ``哥德尔、埃舍尔、巴赫——集异壁之大成''. 商务印书馆 1996. ISBN: 978-7-100-01323-9

\bibitem{Bird97}
Richard Bird, Oege de Moor. ``Algebra of Programming''. University of Oxford, Prentice Hall Europe. 1997. ISBN: 0-13-507245-X.

\bibitem{Gusen2014}
顾森 ``浴缸里的惊叹''. 人民邮电出版社. 2014, ISBN: 9787115355744

% 递归
% ==============================

\bibitem{HanXueTao16}
韩雪涛 ``数学悖论与三次数学危机''. 人民邮电出版社. 2016, ISBN: 9787115430434

\bibitem{StepanovRose15}
[美] 亚历山大 A$\cdot$斯捷潘诺夫,丹尼尔 E$\cdot$罗斯著,爱飞翔译. ``数学与泛型编程:高效编程的奥秘''. 机械工业出版社. 2017, ISBN: 9787111576587

\bibitem{MKlein1972}
[美] 莫里斯$\cdot$克莱因 张理京等译 ``古今数学思想,第一册''. 上海科学技术出版社. 2014, ISBN: 9787547817179

\bibitem{Elements}
[古希腊] 欧几里得 著,兰纪正 朱恩宽 译,梁宗巨 张毓新 徐伯谦 校订 ``几何原本''. 译林出版社. 2014, ISBN: 9787544750066

\bibitem{HanXueTao2009}
韩雪涛 ``好的数学——“下金蛋”的数学问题''. 湖南科学技术出版社. 2009, ISBN: 9787535756725

\bibitem{Bezout-Identity}
Wikipedia ``贝祖等式'' \url{https://en.wikipedia.org/wiki/Bézout's_identity}

\bibitem{LiuXinyu2017}
刘新宇 ``算法新解'' 人民邮电出版社. 2017, ISBN: 9787115440358

\bibitem{wiki-Turing}
Wikipedia ``艾伦$\cdot$图灵'' \url{https://en.wikipedia.org/wiki/Alan_Turing}

\bibitem{Dowek2011}
[法] 吉尔$\cdot$多维克 著,劳佳 译 ``计算进化史:改变数学的命运''. 人民邮电出版社. 2017, ISBN: 9787115447579

\bibitem{SPJ1987}
Simon L. Peyton Jones. ``The implementation of functional programming language''. Prentice Hall. 1987, ISBN: 013453333X

% 抽象代数:群、环、域、伽罗瓦理论
% ==================================

\bibitem{HanXueTao2012}
韩雪涛 ``好的数学——方程的故事''. 湖南科学技术出版社. 2012, ISBN: 9787535770066

\bibitem{Wiki-Galois-theory}
Wikipedia ``伽罗瓦理论''. \url{https://en.wikipedia.org/wiki/Galois_theory}

\bibitem{Wiki-Galois}
Wikipedia ``埃瓦里斯特$\cdot$伽罗瓦''. \url{https://en.wikipedia.org/wiki/Évariste_Galois}

\bibitem{Wiki-Rubik-Cube-group}
Wikipedia ``魔方群''. \url{https://en.wikipedia.org/wiki/Rubik's_Cube_group}

\bibitem{ZhangHeRui1978}
张禾瑞 ``近世代数基础''. 高等教育出版社. 1978, ISBN: 9787040012224

\bibitem{Armstrong1988}
M.A. Armstrong ``群与对称(影印版)''. Springer. 1988. ISBN: 0387966757.

\bibitem{Wiki-Lagrange}
Wikipedia ``约瑟夫$\cdot$拉格朗日''. \url{https://en.wikipedia.org/wiki/Joseph-Louis_Lagrange}

\bibitem{Wiki-FLT-proof}
Wikipedia ``费马小定理的证明''. \url{https://en.wikipedia.org/wiki/Proofs_of_Fermat's_little_theorem}

\bibitem{Wiki-Euler}
Wikipedia ``莱昂哈德$\cdot$欧拉''. \url{https://en.wikipedia.org/wiki/Leonhard_Euler}

\bibitem{Wiki-Carmichael-number}
Wikipedia ``卡米歇尔数''. \url{https://en.wikipedia.org/wiki/Carmichael_number}

\bibitem{Algorithms-DPV}
Sanjoy Dsgupta, Christos Papadimitriou, Umesh Vazirani. 钱枫 邹恒明 注释. ``算法概论(注释版)''. 机械工业出版社. 2009年1月. ISBN: 9787111253617

\bibitem{Wiki-Miller-Rabin}
Wikipedia ``米勒——拉宾素数检验''. \url{https://en.wikipedia.org/wiki/Miller-Rabin_primality_test}

\bibitem{Wiki-Noether}
Wikipedia ``埃米$\cdot$诺特''. \url{https://en.wikipedia.org/wiki/Emmy_Noether}

\bibitem{ZhangPu2013}
章璞. ``伽罗瓦理论:天才的激情''. 高等教育出版社. 2013年5月. ISBN: 9787040372526

\bibitem{Stillwell1994}
John Stillwell. ``Galois Theory for Beginners''. The American Mathematical Monthly, Vol. 101, No. 1 (Jan., 1994), pp. 22-2

\bibitem{Goodman2011}
Dan Goodman. ``An Introduction to Galois Theory''. \url{https://nrich.maths.org/1422}

\bibitem{MArtin}
Michael Artin. ``代数(英文版,第二版)''. 机械工业出版社. 2011年12月. ISBN: 9787111367017

% 范畴论
% =========================================

\bibitem{Dieudonne1987}
[法]让$\cdot$迪厄多内 著,沈用欢 译 ``当代数学,为了人类心智的荣耀''. 上海教育出版社. 2000年3月. ISBN: 7532063062

\bibitem{Monad-Haskell-Wiki}
Haskell Wiki. ``Monad''. \url{https://wiki.haskell.org/Monad}

\bibitem{Wiki-Eilenberg}
Wikipedia. ``塞缪尔$\cdot$艾伦伯格''. \url{https://en.wikipedia.org/wiki/Samuel_Eilenberg}

\bibitem{Wiki-Mac-Lane}
Wikipedia. ``桑德斯$\cdot$麦克兰恩''. \url{https://en.wikipedia.org/wiki/Saunders_Mac_Lane}

\bibitem{Simmons2011}
Harold Simmons. ``An introduction to Category Theory''.  Cambridge University Press; 1 edition, 2011. ISBN: 9780521283045

\bibitem{Wiki-Hoare}
Wikipedia. ``Tony Hoare''. \url{https://en.wikipedia.org/wiki/Tony_Hoare}

\bibitem{Wadler-1989}
Wadler Philip. ``Theorems for free!''. Functional Programming Languages and Computer Architecture, pp. 347-359. Asociation for Computing Machinery. 1989.

\bibitem{Milewski2018}
Bartosz Milewski. ``Category Theory for Programmers''. \url{https://bartoszmilewski.com/2014/10/28/category-theory-for-programmers-the-preface/}

\bibitem{PeterSmith2018}
Peter Smith. ``Category Theory - A Gentle Introduction''. \url{http://www.academia.edu/21694792/A_Gentle_Introduction_to_Category_Theory_Jan_2018_version_}

\bibitem{Wiki-Exponentials}
Wikipedia. ``Exponential Object''. \url{https://en.wikipedia.org/wiki/Exponential_object}

\bibitem{Manes-Arbib-1986}
Manes, E. G. and Arbib, M. A. ``Algebraic Approaches to Program Semantics''. Texts and Monographs in Computer Science. Springer-Verlag. 1986.

\bibitem{Lambek-1968}
Lambek, J. ``A fixpoint theorem for complete categories''. Mathematische Zeischrift, 103, pp.151-161. 1968.

\bibitem{Haskell-foldable}
Wikibooks. ``Haskell/Foldable''. \url{https://en.wikibooks.org/wiki/Haskell/Foldable}

\bibitem{Mac-Lane-1998}
Mac Lane. ``Categories for working mathematicians''. Springer-Verlag. 1998. ISBN: 0387984038.

% 推理
% ===========================================

\bibitem{GLPJ-1993}
Andrew Gill, John Launchbury, Simon L. Peyton Jones. ``A Short Cut to Deforestation''. Functional programming languages and computer architecture. pp. 223-232. 1993.

\bibitem{Bird-2010}
Richard Bird. ``Pearls of Functional Algorithm Design''. Cambridge University Press; 1 edition November 1, 2010. ISBN: 978-0521513388.

\bibitem{Hinze-Harper-James-2010}
Ralf Hinze, Thomas Harper, Daniel W. H. James. ``Theory and Practice of Fusion''. 2010, 22nd international symposium of IFL (Implementation and application of functional languages). pp.19-37.

\bibitem{Takano-Meijer-1995}
Akihiko Takano, Erik Meijer. ``Shortcut Deforestation in Calculational Form''. Functional programming languages and computer architecture. pp. 306-313. 1995.

\bibitem{Knuth-TAOCP-2006}
Donald Knuth. ``The Art of Computer Programming, Volume 4, Fascicle 4: Generating All Trees.'' Reading, MA: Addison-Wesley. ISBN: 978-0321637130. 2006.

% 无穷
% ===========================================
\bibitem{De-linfini-2018}
[法] 让-皮埃尔$\cdot$卢米涅,马克$\cdot$拉雪茨-雷 著,孙展 译. 从无穷开始——科学的困惑与疆界. 人民邮电出版社. 2018. ISBN: 9787115479198

\bibitem{Noguchi2007}
[日] 野口哲也 著,刘慧 韩丽红 译. 数学原来可以这样学. 湖南人民出版社. 2014. ISBN: 9787556100897
% Tetsunori Noguchi. SUGAKUTEKI SENSE GA MINITUKU RENSHUCHO.

\bibitem{Wikipedia-Googol}
Wikipedia. ``Googol''. \url{https://en.wikipedia.org/wiki/Googol}

\bibitem{Wikipedia-Zeno}
Wikipedia. ``Zeno's Paradoxes''. \url{https://en.wikipedia.org/wiki/Zeno's_paradoxes}

\bibitem{GCH}
张锦文,王雪生 ``连续统假设''. 世界数学名题欣赏丛书。辽宁教育出版社 1988. ISBN: 7-5382-0436-9/G$\cdot$445

\bibitem{Courant1969}
[美] R$\cdot$柯朗, H$\cdot$罗宾 著,[英] I$\cdot$斯图尔特 修订,左平,张饴慈 译 ``什么是数学:对思想和方法的基本研究(中文版第四版)''. 上海:复旦大学出版社 2017,ISBN: 9787309086232.

% Richard Courant, Herbert Robbins , Reviewed by Ian Stewart. ``What Is Mathematics? An Elementary Approach to Ideas and Methods 2nd Edition''. Oxford University Press,  1996, ISBN: 978-0195105193.

\bibitem{Poincare1}
[法]彭加勒 著,李醒民 译 ``科学与假设'' 商务印书馆. 2006. ISBN: 978-7-100-04796-8

% 悖论
% ===========================================
\bibitem{Gatys-2015}
Leon A. Gatys, Alexander S. Ecker, Matthias Bethge. ``A Neural Algorithm of Artistic Style.'' 2015. arXiv:1508.06576 [cs.CV] IEEE Conference on Computer Vision and Pattern Recognition (CVPR) 2017.

\bibitem{GuSen-2012}
顾森 《思考的乐趣——Matrix67数学笔记》 人民邮电出版社,2012年,ISBN: 9787115275868

\bibitem{SICP}
Harold Abelson, Gerald Jay Sussman, Julie Sussman 著 裘宗燕 译 ``计算机程序的构造和解释(原书第二版)''. 北京 机械工业出版社 2004年 ISBN: 7-111-13510-5

\bibitem{Poincare2}
[法]彭加勒 著,李醒民 译 ``科学的价值'' 商务印书馆. 2010 ISBN: 978-7-100-07045-4

\bibitem{Ried-1996}
[美]康斯坦丝·瑞德 著,袁向东 / 李文林 译 ``希尔伯特:数学界的亚历山大'' 上海科学技术出版社. 2018-8, ISBN: 978-7-5478-4088-7

% 附录答案
% ========================================

\bibitem{Lockhart2012}
[美] 保罗$\cdot$洛克哈特 著, 王凌云 译. ``度量——一首献给数学的情歌''. 人民邮电出版社. 2015, ISBN: 9787115393180

\end{thebibliography}

\ifx\wholebook\relax \else

\expandafter\enddocument
%\end{document}

\fi
