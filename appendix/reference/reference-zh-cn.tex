\ifx\wholebook\relax \else

\documentclass[UTF8]{article}

%
% loading packages
%

\RequirePackage{ifpdf}
\RequirePackage{ifxetex}

%
%
\ifpdf
  \RequirePackage[pdftex,%
       bookmarksnumbered,%
              colorlinks,%
          linkcolor=blue,%
              hyperindex,%
        plainpages=false,%
       pdfstartview=FitH]{hyperref}
\else\ifxetex
  \RequirePackage[bookmarksnumbered,%
               colorlinks,%
           linkcolor=blue,%
               hyperindex,%
         plainpages=false,%
        pdfstartview=FitH]{hyperref}
\else
  \RequirePackage[dvipdfm,%
        bookmarksnumbered,%
               colorlinks,%
           linkcolor=blue,%
               hyperindex,%
         plainpages=false,%
        pdfstartview=FitH]{hyperref}
\fi\fi
%\usepackage{hyperref}

% other packages
%--------------------------------------------------------------------------
\usepackage{graphicx, color}
\usepackage{wrapfig}
\usepackage{subfig}
\usepackage{multicol}
\usepackage{tikz}
\usetikzlibrary{matrix,positioning,shapes}
\usetikzlibrary{patterns}

\usepackage{amsmath, amsthm, amssymb} % for math
\usepackage{exercise} % for exercise
\usepackage{import} % for nested input

%
% for programming
%
\usepackage{verbatim}
\usepackage{fancyvrb}
\usepackage{listings}
%\usepackage{algorithmic} %old version; we can use algorithmicx instead
%\usepackage[plain]{algorithm} %remove rule (horizontal line on top/below the algorithm
\usepackage{algorithm} %to remove rules change to \usepackage[plain]{algorithm}
%\usepackage{algorithm2e}
\usepackage[noend]{algpseudocode} %for pseudo code, include algorithmicsx automatically
\usepackage{appendix}
\usepackage{makeidx} % for index support
\usepackage{titlesec}
\usepackage{epigraph}

\usepackage[cm-default]{fontspec}
\usepackage{xunicode}
%\usepackage{fontenc}
\usepackage{textcomp}
\usepackage{url}

% detect and select Chinese font
% ------------------------------
% fc-list :lang=zh    % list all Chinese fonts
% fc-list :mono       % list all mono fonts
% fc-cache            % refresh cache to load new installed fonts
\def\macmainfont{STSong}  % Under Mac OS X
\def\macmonofont{Monaco}
\def\winmainfont{SimSun} % Under Windows
\def\winmonofont{Consolas}
\def\linuxmainfont{WenQuanYi Micro Hei} % Under Linux
\def\linuxmainfont{Courier}

\suppressfontnotfounderror1 % Avoid setting exit code (error level) to break make process
\count255=\interactionmode
\batchmode

% main font
\let\mainft=\macmainfont
\font\thefont="\mainft"\space at 10pt
\ifx\thefont\nullfont
  \let\mainft=\winmainfont
  \font\thefont="\mainft"\space at 10pt
  \ifx\the\nullfont
    \let\mainft=\linuxmainfont
    \font\thefont="\mainft"\space at 10pt
    \ifx\the\nullfont
      \errorstopmode
      \errmessage{no suitable Chinese main font found}
    \fi
  \fi
\fi

% mono font
\let\monoft=\macmonofont
\font\thefont="\monoft"\space at 10pt
\ifx\thefont\nullfont
  \let\monoft=\winmonofont
  \font\thefont="\monoft"\space at 10pt
  \ifx\the\nullfont
    \let\monoft=\linuxmonofont
    \font\thefont="\monoft"\space at 10pt
    \ifx\the\nullfont
      \errorstopmode
      \errmessage{no suitable mono font found}
    \fi
  \fi
\fi

\interactionmode=\count255

\setmainfont[Mapping=tex-text]{\mainft}
\setmonofont[Scale=MatchLowercase]{\monoft}   % 英文等宽字体

\XeTeXlinebreaklocale "zh"  % to solve the line breaking issue
\XeTeXlinebreakskip = 0pt plus 1pt minus 0.1pt

\titleformat{\paragraph}
{\normalfont\normalsize\bfseries}{\theparagraph}{1em}{}
\titlespacing*{\paragraph}
{0pt}{3.25ex plus 1ex minus .2ex}{1.5ex plus .2ex}

\lstdefinelanguage{Smalltalk}{
  morekeywords={self,super,true,false,nil,thisContext}, % This is overkill
  morestring=[d]',
  morecomment=[s]{"}{"},
  alsoletter={\#:},
  escapechar={!},
  literate=
    {BANG}{!}1
    {UNDERSCORE}{\_}1
    {\\st}{Smalltalk}9 % convenience -- in case \st occurs in code
    % {'}{{\textquotesingle}}1 % replaced by upquote=true in \lstset
    {_}{{$\leftarrow$}}1
    {>>>}{{\sep}}1
    {^}{{$\uparrow$}}1
    {~}{{$\sim$}}1
    {-}{{\sf -\hspace{-0.13em}-}}1  % the goal is to make - the same width as +
    %{+}{\raisebox{0.08ex}{+}}1		% and to raise + off the baseline to match -
    {-->}{{\quad$\longrightarrow$\quad}}3
	, % Don't forget the comma at the end!
  tabsize=2
}[keywords,comments,strings]

% for literate Haskell code
\lstdefinestyle{Haskell}{
  flexiblecolumns=false,
  basewidth={0.5em,0.45em},
  morecomment=[l]--,
  literate={+}{{$+$}}1 {/}{{$/$}}1 {*}{{$*$}}1 {=}{{$=$}}1
           {>}{{$>$}}1 {<}{{$<$}}1 {\\}{{$\lambda$}}1
           {\\\\}{{\char`\\\char`\\}}1
           {->}{{$\rightarrow$}}2 {>=}{{$\geq$}}2 {<-}{{$\leftarrow$}}2
           {<=}{{$\leq$}}2 {=>}{{$\Rightarrow$}}2
           {\ .}{{$\circ$}}2 {\ .\ }{{$\circ$}}2
           {>>}{{>>}}2 {>>=}{{>>=}}2
           {|}{{$\mid$}}1
}

% "define" Scala
\lstdefinelanguage{Scala}{
  morekeywords={abstract,case,catch,class,def,%
    do,else,extends,false,final,finally,%
    for,if,implicit,import,match,mixin,%
    new,null,object,override,package,%
    private,protected,requires,return,sealed,%
    super,this,throw,trait,true,try,%
    type,val,var,while,with,yield},
  otherkeywords={=>,<-,<\%,<:,>:,\#,@},
  sensitive=true,
  morecomment=[l]{//},
  morecomment=[n]{/*}{*/},
  morestring=[b]",
  morestring=[b]',
  morestring=[b]"""
}

\lstloadlanguages{C, C++, Java, Lisp, Haskell, Python, Smalltalk, Scala}

\lstset{
  basicstyle=\small\ttfamily,
  commentstyle=\rmfamily,
  texcl=true,
  showstringspaces = false,
  upquote=true,
  flexiblecolumns=false
}

\newcommand\doubleplus{+\kern-1.3ex+\kern0.8ex}

% ======================================================================

\def\BibTeX{{\rm B\kern-.05em{\sc i\kern-.025em b}\kern-.08em
    T\kern-.1667em\lower.7ex\hbox{E}\kern-.125emX}}

%
% mathematics
%
\newcommand{\be}{\begin{equation}}
\newcommand{\ee}{\end{equation}}
\newcommand{\bmat}[1]{\left( \begin{array}{#1} }
\newcommand{\emat}{\end{array} \right) }
\newcommand{\VEC}[1]{\mbox{\boldmath $#1$}}

% numbered equation array
\newcommand{\bea}{\begin{eqnarray}}
\newcommand{\eea}{\end{eqnarray}}

% equation array not numbered
\newcommand{\bean}{\begin{eqnarray*}}
\newcommand{\eean}{\end{eqnarray*}}

\newtheorem{theorem}{定理}[section]
\newtheorem{lemma}[theorem]{引理}
\newtheorem{proposition}[theorem]{Proposition}
\newtheorem{corollary}[theorem]{Corollary}

% 中文书籍设置
% ====================================
\renewcommand\contentsname{目\ 录}
%\renewcommand\listfigurename{插图目录}
%\renewcommand\listtablename{表格目录}
\renewcommand\figurename{图}
\renewcommand\tablename{表}
\renewcommand\proofname{证明}
\renewcommand\ExerciseName{练习}
%\renewcommand{\algorithmcfname}{算法}

\ifx\wholebook\relax
\renewcommand\bibname{参\ 考\ 文\ 献}                    %book类型
%\newtheorem{Definition}[Theorem]{定义}
\newtheorem{Theorem}{定理}[chapter]
\newtheorem{example}{例题}[chapter]
\else
\renewcommand\refname{参\ 考\ 文\ 献}
\fi

%\setcounter{secnumdepth}{4}
\titleformat{\chapter}
  {\normalfont\bfseries\Large}
  {第\arabic{chapter}章}
  {12pt}{\Large}
%% \titleformat{\subsection}
%%   {\normalfont\bfseries\large}
%%   {\CJKnumber{\arabic{subsection}}、}
%%   {12pt}{\large}
%% \titleformat{\subsubsection}
%%   {\normalfont\bfseries\normalsize}
%%   {\arabic{subsubsection}.}
%%   {12pt}{\normalsize}

%\renewcommand{\baselinestretch}{1.5}                        %文章行间距为1.5倍。

\makeatletter
\newcommand{\verbatimfont}[1]{\renewcommand{\verbatim@font}{\ttfamily#1}}
\makeatother

\setcounter{tocdepth}{4}
\setcounter{secnumdepth}{4}

%\verbatimfont{\footnotesize}


\setcounter{page}{1}

\begin{document}

\fi

\markboth{参考文献}{编程的数学原理}

\begin{thebibliography}{99}

% 自然数
% =====================================

\bibitem{wiki-number}
Wikipedia. ``古代计数系统的历史''. \url{https://en.wikipedia.org/wiki/History_of_ancient_numeral_systems}

\bibitem{trip-to-number-kindom}
[美]卡尔文$\cdot$C$\cdot$克劳森. ``数学旅行家:漫游数王国''. 袁向东、袁钧译,上海教育出版社。ISBN: 7-5320-7883-3/G $cdot$ 7972

\bibitem{wiki-babylonian-num}
Wikipedia. ``古巴比伦数字''. \url{https://en.wikipedia.org/wiki/Babylonian_numerals}

\bibitem{M-Kline-2007}
[美] M$\cdot$克莱因 著 李宏魁 译 ``数学:确定性的丧失'' 湖南科学技术出版社,2007年4月 ISBN: 978-7-5357-1857-0
% Morris Kline ``Mathematics: The Loss of Certainty''. Oxford University Press, 1980.

\bibitem{GEB}
[美]候世达 ``哥德尔、埃舍尔、巴赫——集异壁之大成''. 商务印书馆 1996. ISBN: 978-7-100-01323-9

\bibitem{Bird97}
Richard Bird, Oege de Moor. ``Algebra of Programming''. University of Oxford, Prentice Hall Europe. 1997. ISBN: 0-13-507245-X.

\bibitem{Gusen2014}
顾森 ``浴缸里的惊叹''. 人民邮电出版社. 2014, ISBN: 9787115355744

% 自然数——附录
% ==============================

\bibitem{Stepanov}
Stepanov and Rose. ``数学与泛型编程''. 爱飞翔译,机械工业出版社。ISBN: 978-7-111-57658-7. 2017. pp147 - 148.

% 递归

\bibitem{HanXueTao16}
韩雪涛 ``数学悖论与三次数学危机''. 人民邮电出版社. 2016, ISBN: 9787115430434

\bibitem{StepanovRose15}
[美] 亚历山大 A$\cdot$斯捷潘诺夫,丹尼尔 E$\cdot$罗斯著,爱飞翔译. ``数学与泛型编程:高效编程的奥秘''. 机械工业出版社. 2017, ISBN: 9787111576587

\bibitem{Elements}
[古希腊] 欧几里得 著,兰纪正 朱恩宽 译,梁宗巨 张毓新 徐伯谦 校订 ``几何原本''. 译林出版社. 2014, ISBN: 9787544750066

\bibitem{HanXueTao2009}
韩雪涛 ``好的数学——“下金蛋”的数学问题''. 湖南科学技术出版社. 2009, ISBN: 9787535756725

\bibitem{Bezout-Identity}
Wikipedia ``贝祖等式'' \url{https://en.wikipedia.org/wiki/Bézout's_identity}

\bibitem{LiuXinyu2017}
刘新宇 ``算法新解'' 人民邮电出版社. 2017, ISBN: 9787115440358

\bibitem{wiki-Turing}
Wikipedia ``艾伦$\cdot$图灵'' \url{https://en.wikipedia.org/wiki/Alan_Turing}

\bibitem{Dowek2011}
[法] 吉尔$\cdot$多维克 著,劳佳 译 ``计算进化史:改变数学的命运''. 人民邮电出版社. 2017, ISBN: 9787115447579

\bibitem{SPJ1987}
Simon L. Peyton Jones. ``The implementation of functional programming language''. Prentice Hall. 1987, ISBN: 013453333X

% 抽象代数:群、环、域、伽罗瓦理论
% ==================================

\bibitem{HanXueTao2012}
韩雪涛 ``好的数学——方程的故事''. 湖南科学技术出版社. 2012, ISBN: 9787535770066

\bibitem{Wiki-Galois-theory}
Wikipedia ``伽罗瓦理论''. \url{https://en.wikipedia.org/wiki/Galois_theory}

\bibitem{Wiki-Galois}
Wikipedia ``埃瓦里斯特$\cdot$伽罗瓦''. \url{https://en.wikipedia.org/wiki/Évariste_Galois}

\bibitem{Wiki-Rubik-Cube-group}
Wikipedia ``魔方群''. \url{https://en.wikipedia.org/wiki/Rubik's_Cube_group}

\bibitem{ZhangHeRui1978}
张禾瑞 ``近世代数基础''. 高等教育出版社. 1978, ISBN: 9787040012224

\bibitem{Armstrong1988}
M.A. Armstrong ``群与对称(影印版)''. Springer. 1988. ISBN: 0387966757.

\bibitem{Wiki-Lagrange}
Wikipedia ``约瑟夫$\cdot$拉格朗日''. \url{https://en.wikipedia.org/wiki/Joseph-Louis_Lagrange}

\bibitem{Wiki-FLT-proof}
Wikipedia ``费马小定理的证明''. \url{https://en.wikipedia.org/wiki/Proofs_of_Fermat's_little_theorem}

\bibitem{Wiki-Euler}
Wikipedia ``莱昂哈德$\cdot$欧拉''. \url{https://en.wikipedia.org/wiki/Leonhard_Euler}

\bibitem{Wiki-Carmichael-number}
Wikipedia ``卡米歇尔数''. \url{https://en.wikipedia.org/wiki/Carmichael_number}

\bibitem{Algorithms-DPV}
Sanjoy Dsgupta, Christos Papadimitriou, Umesh Vazirani. 钱枫 邹恒明 注释. ``算法概论(注释版)''. 机械工业出版社. 2009年1月. ISBN: 9787111253617

\bibitem{Wiki-Miller-Rabin}
Wikipedia ``米勒——拉宾素数检验''. \url{https://en.wikipedia.org/wiki/Miller-Rabin_primality_test}

\bibitem{Wiki-Noether}
Wikipedia ``埃米$\cdot$诺特''. \url{https://en.wikipedia.org/wiki/Emmy_Noether}

\bibitem{ZhangPu2013}
章璞. ``伽罗瓦理论:天才的激情''. 高等教育出版社. 2013年5月. ISBN: 9787040372526

\bibitem{Stillwell1994}
John Stillwell. ``Galois Theory for Beginners''. The American Mathematical Monthly, Vol. 101, No. 1 (Jan., 1994), pp. 22-2

\bibitem{Goodman2011}
Dan Goodman. ``An Introduction to Galois Theory''. \url{https://nrich.maths.org/1422}

\bibitem{MArtin}
Michael Artin. ``代数(英文版,第二版)''. 机械工业出版社. 2011年12月. ISBN: 9787111367017

% 范畴论
% =========================================

\bibitem{Dieudonne1987}
[法]让$\cdot$迪厄多内 著,沈用欢 译 ``当代数学,为了人类心智的荣耀''. 上海教育出版社. 2000年3月. ISBN: 7532063062

\bibitem{Monad-Haskell-Wiki}
Haskell Wiki. ``Monad''. \url{https://wiki.haskell.org/Monad}

\bibitem{Wiki-Eilenberg}
Wikipedia. ``塞缪尔$\cdot$艾伦伯格''. \url{https://en.wikipedia.org/wiki/Samuel_Eilenberg}

\bibitem{Wiki-Mac-Lane}
Wikipedia. ``桑德斯$\cdot$麦克兰恩''. \url{https://en.wikipedia.org/wiki/Saunders_Mac_Lane}

\bibitem{Simmons2011}
Harold Simmons. ``An introduction to Category Theory''.  Cambridge University Press; 1 edition, 2011. ISBN: 9780521283045

\bibitem{Wiki-Hoare}
Wikipedia. ``Tony Hoare''. \url{https://en.wikipedia.org/wiki/Tony_Hoare}

\bibitem{Wadler-1989}
Wadler Philip. ``Theorems for free!''. Functional Programming Languages and Computer Architecture, pp. 347-359. Asociation for Computing Machinery. 1989.

\bibitem{Milewski2018}
Bartosz Milewski. ``Category Theory for Programmers''. \url{https://bartoszmilewski.com/2014/10/28/category-theory-for-programmers-the-preface/}

\bibitem{PeterSmith2018}
Peter Smith. ``Category Theory - A Gentle Introduction''. \url{http://www.academia.edu/21694792/A_Gentle_Introduction_to_Category_Theory_Jan_2018_version_}

\bibitem{Wiki-Exponentials}
Wikipedia. ``Exponential Object''. \url{https://en.wikipedia.org/wiki/Exponential_object}

\bibitem{Manes-Arbib-1986}
Manes, E. G. and Arbib, M. A. ``Algebraic Approaches to Program Semantics''. Texts and Monographs in Computer Science. Springer-Verlag. 1986.

\bibitem{Lambek-1968}
Lambek, J. ``A fixpoint theorem for complete categories''. Mathematische Zeischrift, 103, pp.151-161. 1968.

\bibitem{Haskell-foldable}
Wikibooks. ``Haskell/Foldable''. \url{https://en.wikibooks.org/wiki/Haskell/Foldable}

\bibitem{Mac-Lane-1998}
Mac Lane. ``Categories for working mathematicians''. Springer-Verlag. 1998. ISBN: 0387984038.

% 推理
% ===========================================

\bibitem{GLPJ-1993}
Andrew Gill, John Launchbury, Simon L. Peyton Jones. ``A Short Cut to Deforestation''. Functional programming languages and computer architecture. pp. 223-232. 1993.

\bibitem{Bird-2010}
Richard Bird. ``Pearls of Functional Algorithm Design''. Cambridge University Press; 1 edition November 1, 2010. ISBN: 978-0521513388.

\bibitem{Hinze-Harper-James-2010}
Ralf Hinze, Thomas Harper, Daniel W. H. James. ``Theory and Practice of Fusion''. 2010, 22nd international symposium of IFL (Implementation and application of functional languages). pp.19-37.

\bibitem{Takano-Meijer-1995}
Akihiko Takano, Erik Meijer. ``Shortcut Deforestation in Calculational Form''. Functional programming languages and computer architecture. pp. 306-313. 1995.

\bibitem{Knuth-TAOCP-2006}
Donald Knuth. ``The Art of Computer Programming, Volume 4, Fascicle 4: Generating All Trees.'' Reading, MA: Addison-Wesley. ISBN: 978-0321637130. 2006.

% 无穷
% ===========================================
\bibitem{De-linfini-2018}
[法] 让-皮埃尔$\cdot$卢米涅,马克$\cdot$拉雪茨-雷 著,孙展 译. 从无穷开始——科学的困惑与疆界. 人民邮电出版社. 2018. ISBN: 9787115479198

\bibitem{Noguchi2007}
[日] 野口哲也 著,刘慧 韩丽红 译. 数学原来可以这样学. 湖南人民出版社. 2014. ISBN: 9787556100897
% Tetsunori Noguchi. SUGAKUTEKI SENSE GA MINITUKU RENSHUCHO.

\bibitem{Wikipedia-Googol}
Wikipedia. ``Googol''. \url{https://en.wikipedia.org/wiki/Googol}

\bibitem{Wikipedia-Zeno}
Wikipedia. ``Zeno's Paradoxes''. \url{https://en.wikipedia.org/wiki/Zeno's_paradoxes}

\bibitem{GCH}
张锦文,王雪生 ``连续统假设''. 世界数学名题欣赏丛书。辽宁教育出版社 1988. ISBN: 7-5382-0436-9/G$\cdot$445

\end{thebibliography}

\expandafter\enddocument
%\end{document}

\fi
