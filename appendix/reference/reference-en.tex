\ifx\wholebook\relax \else

\documentclass{article}
\usepackage[nomarginpar
  %, margin=.5in
]{geometry}

\addtolength{\oddsidemargin}{-0.05in}
\addtolength{\evensidemargin}{-0.05in}
\addtolength{\textwidth}{0.1in}

\usepackage[en]{../../prelude}

\setcounter{page}{1}

\begin{document}

\fi

\markboth{Reference}{Mathematics of Programming}

\phantomsection  % so hyperref creates bookmarks
\addcontentsline{toc}{chapter}{Reference}

\begin{thebibliography}{99}

% Natural numbers
% =====================================

\bibitem{wiki-number}
Wikipedia. ``History of ancient numeral systems''. \url{https://en.wikipedia.org/wiki/History_of_ancient_numeral_systems}

\bibitem{Calvin-Clawson-1994}
Calvin C Clawson. ``The Mathematical Traveler, Exploring the Grand History of Numbers''. Springer. 1994, ISBN: 9780306446450

\bibitem{wiki-babylonian-num}
Wikipedia. ``Babylonian numerals''. \url{https://en.wikipedia.org/wiki/Babylonian_numerals}

\bibitem{M-Kline-2007}
Morris Kline ``Mathematics: The Loss of Certainty''. Oxford University Press, 1980. 0-19-502754-X.

\bibitem{GEB}
Douglas R. Hofstadter ``Gödel, Escher, Bach: An Eternal Golden Braid''. Basic Books; Anniversary edition (February 5, 1999). ISBN: 978-0465026562

\bibitem{Bird97}
Richard Bird, Oege de Moor. ``Algebra of Programming''. University of Oxford, Prentice Hall Europe. 1997. ISBN: 0-13-507245-X.

\bibitem{Gusen2014}
{\fontspec{\cnmainft}顾森 《浴缸里的惊叹》} People's postal Press. 2014, ISBN: 9787115355744

% recursion
% ==============================

\bibitem{HanXueTao16}
{\fontspec{\cnmainft}韩雪涛 ``数学悖论与三次数学危机''. 人民邮电出版社. 2016, ISBN: 9787115430434}

\bibitem{MKlein1972}
Morris Klein. ``Mathematical thought from Ancient to Modern Times, Vol. 1''. Oxford University Press; 1972. ISBN: 9780195061352

\bibitem{StepanovRose15}
Alexander A. Stepanov, Daniel E. Rose ``From Mathematics to Generic Programming''. Addison-Wesley Professional; 1 edition (November 17, 2014) ISBN-13: 978-0321942043

\bibitem{Elements}
Euclid, Thomas Heath (Translator) ``Euclid's Elements (The Thirteen Books) ''. Digireads.com Publishing (December 26, 2017). ISBN-13: 978-1420956474

\bibitem{HanXueTao2009}
{\fontspec{\cnmainft}韩雪涛 ``好的数学——“下金蛋”的数学问题''. 湖南科学技术出版社. 2009, ISBN: 9787535756725}

\bibitem{Bezout-Identity}
Wikipedia ``Bézout's identity'' \url{https://en.wikipedia.org/wiki/Bézout's_identity}

\bibitem{LiuXinyu2017}
{\fontspec{\cnmainft}刘新宇 ``算法新解'' 人民邮电出版社. 2017, ISBN: 9787115440358}

\bibitem{wiki-Turing}
Wikipedia ``Alan Turing'' \url{https://en.wikipedia.org/wiki/Alan_Turing}

\bibitem{Dowek2011}
by Serge Abiteboul (Author), Gilles Dowek (Author), K-Rae Nelson (Translator) ``The Age of Algorithms''. Cambridge University Press (December 31, 2019) ISBN-13: 978-1108745420

\bibitem{SPJ1987}
Simon L. Peyton Jones. ``The implementation of functional programming language''. Prentice Hall. 1987, ISBN: 013453333X

% Abstract algebra
% ==================================

\bibitem{HanXueTao2012}
{\fontspec{\cnmainft}韩雪涛 ``好的数学——方程的故事''. 湖南科学技术出版社.} 2012, ISBN: 9787535770066

\bibitem{Wiki-Galois-theory}
Wikipedia ``Galois theory''. \url{https://en.wikipedia.org/wiki/Galois_theory}

\bibitem{Wiki-Galois}
Wikipedia ``Évariste Galois''. \url{https://en.wikipedia.org/wiki/Évariste_Galois}

\bibitem{Galois-1832}
Anita R Singh. ``The Last Mathematical Testament of Galois'' Resonance Oct 1999. pp93-100.

\bibitem{Liouville-1846}
Sawilowsky, Shlomo S. and Cuzzocrea, John L. (2005) ``Joseph Liouville’s `Mathematical Works Of Évariste Galois','' Journal of Modern Applied Statistical Methods: Vol. 5 : Iss. 2 , Article 32. DOI: 10.22237/jmasm/1162355460

\bibitem{Wiki-Rubik-Cube-group}
Wikipedia ``Rubik's Cube group''. \url{https://en.wikipedia.org/wiki/Rubik's_Cube_group}

\bibitem{ZhangHeRui1978}
{\fontspec{\cnmainft}张禾瑞 ``近世代数基础''. 高等教育出版社.} 1978, ISBN: 9787040012224

\bibitem{Armstrong1988}
M.A. Armstrong ``Groups and Symmetry''. Springer. 1988. ISBN: 0387966757.

\bibitem{Wiki-Lagrange}
Wikipedia ``Joseph-Louis Lagrange''. \url{https://en.wikipedia.org/wiki/Joseph-Louis_Lagrange}

\bibitem{Wiki-FLT-proof}
Wikipedia ``Proofs of Fermat's little theorem''. \url{https://en.wikipedia.org/wiki/Proofs_of_Fermat's_little_theorem}

\bibitem{Wiki-Euler}
Wikipedia ``Leonhard Euler''. \url{https://en.wikipedia.org/wiki/Leonhard_Euler}

\bibitem{Wiki-Carmichael-number}
Wikipedia ``Carmichaeal number''. \url{https://en.wikipedia.org/wiki/Carmichael_number}

\bibitem{Algorithms-DPV}
Sanjoy Dsgupta, Christos Papadimitriou, Umesh Vazirani. ``Algorithms''. McGraw-Hill Education. Sept. 2006. ISBN: 9780073523408

\bibitem{Wiki-Miller-Rabin}
Wikipedia ``Miller-Rabin primality test''. \url{https://en.wikipedia.org/wiki/Miller-Rabin_primality_test}

\bibitem{Wiki-Noether}
Wikipedia ``Emmy Noether''. \url{https://en.wikipedia.org/wiki/Emmy_Noether}

\bibitem{ZhangPu2013}
{\fontspec{\cnmainft}章璞. ``伽罗瓦理论:天才的激情''. 高等教育出版社.} 2013年5月. ISBN: 9787040372526

\bibitem{Stillwell1994}
John Stillwell. ``Galois Theory for Beginners''. The American Mathematical Monthly, Vol. 101, No. 1 (Jan., 1994), pp. 22-2

\bibitem{Goodman2011}
Dan Goodman. ``An Introduction to Galois Theory''. \url{https://nrich.maths.org/1422}

\bibitem{MArtin}
Michael Artin. ``Algebra (Second Edition)''. Pearson. Feb. 2017. ISBN: 9780134689609

\bibitem{Weyl1952}
Hermann Weyl. ``Symmetry''. Princeton University Press; Reprint edition (October 4, 2016). ISBN: 978-0691173252


% Category theory
% =========================================

\bibitem{Dieudonne1987}
Jean Dieudonne. ``Mathematics — The Music of Reason''. Springer Science and Business Media, Jul 20, 1998. ISBN: 9783540533467

\bibitem{Monad-Haskell-Wiki}
Haskell Wiki. ``Monad''. \url{https://wiki.haskell.org/Monad}

\bibitem{Wiki-Eilenberg}
Wikipedia. ``Samuel Eilenberg''. \url{https://en.wikipedia.org/wiki/Samuel_Eilenberg}

\bibitem{Wiki-Mac-Lane}
Wikipedia. ``Saunders Mac Lane''. \url{https://en.wikipedia.org/wiki/Saunders_Mac_Lane}

\bibitem{Simmons2011}
Harold Simmons. ``An introduction to Category Theory''.  Cambridge University Press; 1 edition, 2011. ISBN: 9780521283045

\bibitem{Wiki-Hoare}
Wikipedia. ``Tony Hoare''. \url{https://en.wikipedia.org/wiki/Tony_Hoare}

\bibitem{Wadler-1989}
Wadler Philip. ``Theorems for free!''. Functional Programming Languages and Computer Architecture, pp. 347-359. Asociation for Computing Machinery. 1989.

\bibitem{Milewski2018}
Bartosz Milewski. ``Category Theory for Programmers''. \url{https://bartoszmilewski.com/2014/10/28/category-theory-for-programmers-the-preface/}

\bibitem{PeterSmith2018}
Peter Smith. ``Category Theory - A Gentle Introduction''. \url{http://www.academia.edu/21694792/A_Gentle_Introduction_to_Category_Theory_Jan_2018_version_}

\bibitem{Wiki-Exponentials}
Wikipedia. ``Exponential Object''. \url{https://en.wikipedia.org/wiki/Exponential_object}

\bibitem{Manes-Arbib-1986}
Manes, E. G. and Arbib, M. A. ``Algebraic Approaches to Program Semantics''. Texts and Monographs in Computer Science. Springer-Verlag. 1986.

\bibitem{Lambek-1968}
Lambek, J. ``A fixpoint theorem for complete categories''. Mathematische Zeischrift, 103, pp.151-161. 1968.

\bibitem{Haskell-foldable}
Wikibooks. ``Haskell/Foldable''. \url{https://en.wikibooks.org/wiki/Haskell/Foldable}

\bibitem{Mac-Lane-1998}
Mac Lane. ``Categories for working mathematicians''. Springer-Verlag. 1998. ISBN: 0387984038.

% 推理
% ===========================================

\bibitem{GLPJ-1993}
Andrew Gill, John Launchbury, Simon L. Peyton Jones. ``A Short Cut to Deforestation''. Functional programming languages and computer architecture. pp. 223-232. 1993.

\bibitem{Bird-2010}
Richard Bird. ``Pearls of Functional Algorithm Design''. Cambridge University Press; 1 edition November 1, 2010. ISBN: 978-0521513388.

\bibitem{Hinze-Harper-James-2010}
Ralf Hinze, Thomas Harper, Daniel W. H. James. ``Theory and Practice of Fusion''. 2010, 22nd international symposium of IFL (Implementation and application of functional languages). pp.19-37.

\bibitem{Takano-Meijer-1995}
Akihiko Takano, Erik Meijer. ``Shortcut Deforestation in Calculational Form''. Functional programming languages and computer architecture. pp. 306-313. 1995.

\bibitem{Knuth-TAOCP-2006}
Donald Knuth. ``The Art of Computer Programming, Volume 4, Fascicle 4: Generating All Trees.'' Reading, MA: Addison-Wesley. ISBN: 978-0321637130. 2006.

% 无穷
% ===========================================
\bibitem{De-linfini-2018}
Jean-Pierre Luminet, Marc Lachièze-Rey. ``De l'infini - Horizons cosmiques, multivers et vide quantique''. DUNOD, 2016. ISBN: 9782100738380

\bibitem{Noguchi2007}
{\fontspec{\cnmainft}野口哲典. ``数学的センスが身につく練習帳''. ソフトバンククリエイティブ} 2007. ISBN: 9784797339314

\bibitem{Wikipedia-Googol}
Wikipedia. ``Googol''. \url{https://en.wikipedia.org/wiki/Googol}

\bibitem{Wikipedia-Zeno}
Wikipedia. ``Zeno's Paradoxes''. \url{https://en.wikipedia.org/wiki/Zeno's_paradoxes}

\bibitem{GCH}
{\fontspec{\cnmainft}张锦文,王雪生 ``连续统假设''. 世界数学名题欣赏丛书。辽宁教育出版社.} 1988. ISBN: 7-5382-0436-9/G$\cdot$445

\bibitem{Courant1969}
Richard Courant, Herbert Robbins, Reviewed by Ian Stewart. ``What Is Mathematics? An Elementary Approach to Ideas and Methods 2nd Edition''. Oxford University Press,  1996, ISBN: 978-0195105193.

\bibitem{Poincare1}
Henri Poincaré. ``Science and Hypothesis''. Franklin Classics. 2018. ISBN: 9780342349418

% 悖论
% ===========================================
\bibitem{Gatys-2015}
Leon A. Gatys, Alexander S. Ecker, Matthias Bethge. ``A Neural Algorithm of Artistic Style.'' 2015. arXiv:1508.06576 [cs.CV] IEEE Conference on Computer Vision and Pattern Recognition (CVPR) 2017.

\bibitem{GuSen-2012}
{\fontspec{\cnmainft}顾森 《思考的乐趣——Matrix67数学笔记》 人民邮电出版社.} 2012, ISBN: 9787115275868

\bibitem{SICP}
Harold Abelson, Gerald Jay Sussman, Julie Sussman. ``Structure and Interpretation of Computer Programs''. MIT Press, 1984; ISBN 0-262-01077-1

\bibitem{Poincare2}
Henri Poincaré. ``The Value of Science''. Modern Library, 2001. ISBN: 978-0375758485

\bibitem{Ried-1996}
Constance Ried. ``Hilbert''. Springer, 1st Printing edition, 1996, ISBN: 978-0387946740

% 附录答案
% ========================================

\bibitem{Lockhart2012}
Paul Lockhart. ``Measurement''. Belknap Press: An Imprint of Harvard University Press; Reprint edition 2014, ISBN: 978-0674284388

\end{thebibliography}

\ifx\wholebook\relax \else

\expandafter\enddocument
%\end{document}

\fi
