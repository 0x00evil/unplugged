\ifx\wholebook\relax \else

\documentclass[UTF8]{article}

\usepackage[cn]{../../prelude}

\setcounter{page}{1}

\begin{document}

\title{参考答案}

\author{刘新宇
\thanks{{\bfseries 刘新宇} \newline
  Email: liuxinyu95@gmail.com \newline}
  }

\maketitle
\fi

\markboth{参考答案}{编程中的数学}

\chapter*{参考答案}
\phantomsection  % so hyperref creates bookmarks
\addcontentsline{toc}{chapter}{参考答案}

\section{前言}

\begin{enumerate}
\item 编程实现一个井字棋游戏是传统人工智能中的经典问题,而计算机可以轻松算出三个数字的和并判断其是否等于15。请利用这个同构编写一个简化的井字棋程序,并做到不被人类玩家击败。
\end{enumerate}

\section{自然数}

\begin{enumerate}
\item 定义0的后继为1,证明对于任何自然数都有$a \cdot 1 = a$

首先用数学归纳法证明$0 + a = a$这个结论,见附录I。然后:
\[
\begin{array}{rlr}
a' \cdot 1 & = a' \cdot 0' & \text{定义0的后继为1} \\
           & = a' \cdot 0 + a' & \text{乘法定义规则二} \\
           & = 0 + a' & \text{乘法定义规则一} \\
           & = a' & \text{此前证明的结论}
\end{array}
\]

\item 证明乘法分配律

可以用数学归纳法证明左侧的分配律$c(a + b) = ca + cb$。首先是$b = 0$的情况:

\bre
c(a + 0) & = & ca & \text{加法规则一} \\
         & = & ca + 0 & \text{反向用加法规则一} \\
         & = & ca + c0 & \text{反向用乘法规则一} \\
\ere

递推假设$c(a + b) = ca + cb$,接下来证明$c(a + b') = ca + cb'$

\bre
c(a + b') & = & c(a + b)' & \text{加法规则二} \\
          & = & c(a + b) + c & \text{乘法规则二} \\
          & = & ca + cb + c & \text{递推假设} \\
          & = & ca + (cb + c) & \text{加法结合律} \\
          & = & ca + cb' & \text{反向用乘法规则二} \\
\ere

\item 证明乘法结合律和交换律

我们只证明乘法结合律$(ab)c = a(bc)$,乘法交换律的证明则给出一个提纲。利用数学归纳法,首先是$c = 0$的情况:

\bre
(ab)0 & = & 0 & \text{乘法规则一} \\
      & = & a0 & \text{反向用乘法规则一} \\
      & = & a(b0) & \text{反向用乘法规则一} \\
\ere

递推假设$(ab)c = a(bc)$,接下来要证明$(ab)c' = a(bc')$

\bre
(ab)c' & = & (ab)c + ab & \text{乘法规则二} \\
       & = & a(bc) + ab & \text{递推假设} \\
       & = & a(bc + b) & \text{上题证明的分配律} \\
       & = & a(bc') & \text{反向用乘法规则二} \\
\ere

证明乘法交换律可以分为三步,都使用数学归纳法。首先证明$1a = a$,然后再证明右侧的分配律$(a + b)c = ac + bc$,最后再证明交换律。

\item 如何利用皮亚诺公里验证3 + 147 = 150

我们先看看经典的2 + 2 = 4是怎么证明的:

\bre
2 + 2 & = & 0'' + 0'' & \text{2是0的两次后继} \\
      & = & (0'' + 0')' & \text{加法定义规则二} \\
      & = & ((0'' + 0)')' & \text{加法定义规则二} \\
      & = & ((0'')')' & \text{加法定义规则一} \\
      & = & 0'''' = 4 & \text{0的4次后继} \\
\ere

显然用这个方法证明3 + 147 = 150的话太冗长了,我们可以用先前证明的加法交换律证明147 + 3 = 150会容易一些。另一个方法是通过数学归纳法证明$3 + a = a'''$。

\item 使用$foldn$定义平方$()^2$。

可以利用递推关系$(n+1)^2 = n^2 + 2n + 1$来定义平方:

\[
()^2 = 2nd \cdot foldn\ (0, 0)\ h
\]

其中$h$接受一对值$(i, s)$,分别代表自然数$i$和它的平方$s$。它将第一个值递增1,然后利用平方展开式求出下一个平方数。

\[
h\ (i, s) = (i + 1, s + 2i + 1)
\]

\item 使用$foldn$定义$()^m$,计算给定自然数的$m$次幂。

\item 使用$foldn$定义奇数的和。它会产生怎样的序列?
\item 地面上有一排洞,一只狐狸藏在某个洞中。每天狐狸会移动到相邻的下一个洞里。如果每天只能检查一个洞,请给出一个捉到狐狸的策略,并证明这个策略有效。如果狐狸每天移动的不止一个洞呢?

\end{enumerate}

\section{递归}

\section{}

\ifx\wholebook\relax \else

\expandafter\enddocument
%\end{document}

\fi
