\ifx\wholebook\relax \else

\documentclass[UTF8]{article}

\usepackage[nomarginpar
  %, margin=.5in
]{geometry}

\addtolength{\oddsidemargin}{-0.05in}
\addtolength{\evensidemargin}{-0.05in}
\addtolength{\textwidth}{0.1in}

\usepackage[cn]{../../prelude}

\setcounter{page}{1}

\begin{document}

\title{参考答案}

\author{刘新宇
\thanks{{\bfseries 刘新宇} \newline
  Email: liuxinyu95@gmail.com \newline}
  }

\maketitle
\fi

\markboth{参考答案}{编程中的数学}

\chapter*{参考答案}
\phantomsection  % so hyperref creates bookmarks
\addcontentsline{toc}{chapter}{参考答案}

\shipoutAnswer

\section{范畴}

\begin{enumerate}

\item {请使用叠加操作$foldr$来定义列表函子的箭头映射。}

本质上就是用$foldr$定义列表的逐一映射:

\[
fmap\ f = foldr\ f\ Nil
\]

\item {证明可能函子和列表函子的组合$\mathbf{Maybe} \circ \mathbf{List}$与$\mathbf{List} \circ \mathbf{Maybe}$仍然是函子。}

我们只证明$\mathbf{Maybe} \circ \mathbf{List}$仍然是函子。另一证明与此类似。任何对象$A$被映射为$\mathbf{Maybe} (\mathbf{List}\ A)$。对于箭头,我们先看恒等箭头的情况:

\bre
(\mathbf{Maybe} \circ \mathbf{List})\ id & = & \mathbf{Maybe} (\mathbf{List}\ id) & \text{函子的组合} \\
 & = & \mathbf{Maybe}\ id & \text{列表函子的恒等箭头} \\
 & = & id & \text{可能函子的恒等箭头} \\
\ere

然后是箭头组合:

\[
\resizebox{\textwidth}{!}{\ensuremath{
\begin{array}{rcll}
(\mathbf{Maybe} \circ \mathbf{List})\ (f \circ g) & = & \mathbf{Maybe} (\mathbf{List}\ (f \circ g)) & \text{函子的组合} \\
 & = & \mathbf{Maybe} ((\mathbf{List}\ f) \circ (\mathbf{List}\ g)) & \text{列表函子的箭头组合性质} \\
 & = & (\mathbf{Maybe}\ (\mathbf{List}\ f)) \circ (\mathbf{Maybe}\ (\mathbf{List}\ g)) & \text{可能函子的箭头组合性质} \\
 & = & ((\mathbf{Maybe} \circ \mathbf{List})\ f) \circ ((\mathbf{Maybe} \circ \mathbf{List})\ g) & \text{函子的组合} \\
\end{array}
}}
\]

\item {证明任意函子的组合$\mathbf{G} \circ \mathbf{F}$仍然是函子。}

仿照前一题,我们分别证明函子的组合满足恒等箭头,和箭头组合的性质即可。首先是恒等箭头性质:

\bre
(\mathbf{G} \circ \mathbf{F})\ id & = & \mathbf{G} (\mathbf{F}\ id) & \text{函子的组合} \\
 & = & \mathbf{G}\ id & \text{函子$\mathbf{F}$的恒等箭头} \\
 & = & id & \text{函子$\mathbf{G}$的恒等箭头} \\
\ere

然后是箭头组合:

\bre
(\mathbf{G} \circ \mathbf{F})\ (\phi \circ \psi) & = & \mathbf{G} (\mathbf{F}\ (\phi \circ \psi)) & \text{函子的组合} \\
 & = & \mathbf{G} ((\mathbf{F}\ \phi) \circ (\mathbf{F}\ \psi)) & \text{函子$\mathbf{F}$的箭头组合性质} \\
 & = & (\mathbf{G}\ (\mathbf{F}\ \phi)) \circ (\mathbf{G}\ (\mathbf{F}\ \psi)) & \text{函子$\mathbf{G}$的箭头组合性质} \\
 & = & ((\mathbf{G} \circ \mathbf{F})\ \phi) \circ ((\mathbf{G} \circ \mathbf{F})\ \psi) & \text{函子的组合} \\
\ere

\item {思考一个预序集范畴上的函子的例子。}

预序集范畴上的函子是一个单调映射。
% monotone function(map)

\item {回顾第二章中介绍的二叉树,请定义一个二叉树函子。}

考虑集合全函数范畴上的对象$A$,二叉树函子将其映射为:

\lstset{language=Haskell, frame=none}
\begin{lstlisting}
data Tree A = Empty | Branch (Tree A) A (Tree A)
\end{lstlisting}

对于箭头$A \arrowto{f} B$,二叉树函子将其映射为:

\bre
fmap\ f\ Empty & = & Empty \\
fmap\ f\ (Branch\ l\ x\ r) & = & Branch\ (fmap\ f\ l)\ (f\ x)\ (fmap\ f\ r) \\
\ere

或者利用第二章定义的$mapt$:

\[
fmap = mapt
\]

\item {考虑偏序集(poset)中的两个对象,它们的积是什么?和是什么?}

第三章中,我们说到一个偏序集本身就是一个范畴,每个元素都是一个对象,任何两个对象间最多有一个箭头(如果有序关系,则存在箭头)。对于偏序集中的两个元素(对象)$a$和$b$,如果它们都有指向上下游的箭头,则

\[
\text{交运算meet}\ a \land b \quad \quad \quad \text{并运算join}\ a \lor b
\]

是这一对对象的

\[
\text{积} \quad \quad \quad \text{和}
\]

其中交运算是两个对象的最小上界,而并运算是两个对象的最大下界。由于最小上界和最大下界并不一定存在,所以偏序集中任何两个对象的积与和也并不一定存在。

% https://en.wikipedia.org/wiki/Join_and_meet
%In mathematics, specifically order theory, the join and meet of a subset S of a partially ordered set P are respectively the supremum (least upper bound) of S, denoted ⋁S, and infimum (greatest lower bound) of S, denoted ⋀S. In general, the join and meet of a subset of a partially ordered set need not exist. Join and meet are dual to one another with respect to order inversion.

%\item {考虑集合范畴$\pmb{Set}$,积中的两个箭头$fst, snd$是右消去(epic)的么?和中的两个箭头$left, right$是左消去(monic)的么?}
\item {证明和的吸收率,并验证和函子的组合性质。}

和的吸收率是:

\[
[p, q] \circ (f + g) = [p \circ f, q \circ g]
\]

\begin{proof}
\blre
  & [p, q] \circ (f + g)  \\
= & [p, q] \circ [left \circ f, right \circ g] & \text{$+$的定义} \\
= & [[p, q] \circ (left \circ f), [p, q] \circ (right \circ g)] & \text{和的融合律} \\
= & [[p, q] \circ left \circ f, [p, q] \circ right \circ g] & \text{结合性} \\
= & [p \circ f, q \circ g] & \text{和的消去律} \\
\elre
\end{proof}

和函子的组合性质:

\[
 (f + g) \circ (f' + g') = f \circ f' + g \circ g'
\]

\begin{proof}
令$p = left \circ f$,$q = right \circ g$

\blre
  & (f + g) \circ (f' + g') \\
= & [left \circ f, right \circ g] \circ (f' + g') & \text{$+$的定义} \\
= & [p, q] \circ (f' + g') & \text{用$p, q$代换} \\
= & [p \circ f', q \circ g'] & \text{和的吸收律} \\
= & [left \circ f \circ f', right \circ g \circ g'] & \text{把$p, q$代回} \\
= & [left \circ (f \circ f'), right \circ (g \circ g')] & \text{结合律} \\
= & f \circ f' + g \circ g' & \text{反向用$+$的定义} \\
\elre
\end{proof}

\item {证明$swap$满足自然变换的条件$(g \times f) \circ swap = swap \circ (f \times g)$}

对于$A \arrowto{f} C$和$B \arrowto{g} D$,我们要证明下面的范畴图可交换。

\begin{center}
\begin{tikzpicture}
  \matrix (m) [matrix of math nodes,
               row sep=3em, column sep=5em, minimum width=2em]{
     (A, B) & (B, A) \\
     (C, D) & (D, C) \\};
  \path[-stealth]
    (m-1-1) edge node [above] {$swap_{A, B}$} (m-1-2)
    (m-2-1) edge node [below] {$swap_{C, D}$} (m-2-2)
    (m-1-1) edge node [left] {$f \times g$} (m-2-1)
    (m-1-2) edge node [right] {$g \times f = swap\ f \times g$} (m-2-2);
\end{tikzpicture}
\end{center}

\begin{proof}
\blre
  & ((g \times f) \circ swap)\ (A, B) \\
= & (g \times f)\ (swap\ (A, B)) & \text{组合的定义} \\
= & (g \times f) \circ (B, A) & \text{$swap$的定义} \\
= & (g\ B, f\ A) & \text{箭头的积} \\
= & (D, C) & \text{箭头$g, f$的定义} \\
= & swap\ (C, D) & \text{反向用$swap$的定义} \\
= & swap\ (f\ A, g\ B) & \text{反向用$f, g$的定义} \\
= & swap\ ((f \times g)\ (A, B)) & \text{箭头的积} \\
= & (swap \circ (f \times g))\ (A, B) & \text{反向用组合的定义} \\
\elre
\end{proof}

\item {证明多态函数$length$是一个自然变换,其定义如下:
\[
\begin{array}{l}
length : [A] \to Int \\
length\ [] = 0 \\
length\ (x:xs) = 1 + length\ xs
\end{array}
\]
}

对于任何对象$A$,它的索引箭头$length$为:

\[
[A] \arrowto{length_A} \mathbf{K}_{Int}\ A
\]

其中$\mathbf{K}_{Int}$是常函子,它将任何对象映射到$Int$,所有箭头映射为恒等箭头$id_{int}$。对于箭头$A \arrowto{f} B$,我们要证明下面的范畴图可交换。

\begin{center}
\begin{tikzpicture}
  \matrix (m) [matrix of math nodes,
               row sep=3em, column sep=5em, minimum width=2em]{
     A & \lbrack A \rbrack & \mathbf{K}_{Int}\ A \\
     B & \lbrack B \rbrack & \mathbf{K}_{Int}\ B \\};
  \path[-stealth]
    (m-1-1) edge node [left] {$f$} (m-2-1)
    % square
    (m-1-2) edge node [above] {$length_A$} (m-1-3)
    (m-2-2) edge node [below] {$length_B$} (m-2-3)
    (m-1-2) edge node [left] {$\mathbf{List}(f)$} (m-2-2)
    (m-1-3) edge node [right] {$\mathbf{K}_{Int}(f)$} (m-2-3);
\end{tikzpicture}
\end{center}

利用常函子的定义,这个范畴图等价于:

\begin{center}
\begin{tikzpicture}
  \matrix (m) [matrix of math nodes,
               row sep=3em, column sep=5em, minimum width=2em]{
     A & \lbrack A \rbrack & Int \\
     B & \lbrack B \rbrack & Int \\};
  \path[-stealth]
    (m-1-1) edge node [left] {$f$} (m-2-1)
    % square
    (m-1-2) edge node [above] {$length_A$} (m-1-3)
    (m-2-2) edge node [below] {$length_B$} (m-2-3)
    (m-1-2) edge node [left] {$\mathbf{List}(f)$} (m-2-2)
    (m-1-3) edge node [right] {$id$} (m-2-3);
\end{tikzpicture}
\end{center}

也就是要证明:

\[
id \circ length_{A} = length_{B} \circ \mathbf{List}(f)
\]

即:$length_{A} = length_{B} \circ \mathbf{List}(f)$


\begin{proof}
我们用数学归纳法证明。首先是空列表:

\blre
  & length_B \circ \mathbf{List}(f) [] \\
= & length_B [] & \text{列表函子的定义} \\
= & 0 & \text{$length$的定义} \\
= & length_A\ [] & \text{反向用$length$的定义}
\elre

然后是归纳假设,设$length_B \circ \mathbf{List}(f)\ as = length_A\ as$成立,我们有:

\blre
  & length_B \circ \mathbf{List}(f) (a:as) \\
= & length_B\ (f(a) : \mathbf{List}(f)\ as) & \text{列表函子的定义} \\
= & 1 + length_B\ (\mathbf{List}(f)\ as) & \text{$length$的定义} \\
= & 1 + length_B\ \circ \mathbf{List}(f)\ as & \text{箭头组合} \\
= & 1 + length_A\ as & \text{归纳假设} \\
= & length_A\ (a:as) & \text{反向用$length$的定义}
\elre
\end{proof}

\item {自然变换也可以进行组合,考虑两个自然变换$\mathbf{F} \arrowto{\phi} \mathbf{G}$和$\mathbf{G} \arrowto{\psi} \mathbf{H}$,对于任意箭头$A \arrowto{f} B$,试画出自然变换组合$\phi \circ \psi$的范畴图,并列出可交换性的条件。}

\begin{center}
\begin{tikzpicture}
  \matrix (m) [matrix of math nodes,
               row sep=3em, column sep=5em, minimum width=2em]{
     \mathbf{F}A & \mathbf{G}A & \mathbf{H}A \\
     \mathbf{F}B & \mathbf{G}B & \mathbf{H}B \\};
  \path[-stealth]
    (m-1-1) edge node [left] {$\mathbf{F}(f)$} (m-2-1)
    (m-1-1) edge node [above] {$\phi_A$} (m-1-2)
    (m-2-1) edge node [below] {$\phi_B$} (m-2-2)
    (m-1-2) edge node [above] {$\psi_A$} (m-1-3)
    (m-2-2) edge node [below] {$\psi_B$} (m-2-3)
    (m-1-2) edge node [left] {$\mathbf{G}(f)$} (m-2-2)
    (m-1-3) edge node [right] {$\mathbf{H}(f)$} (m-2-3);
\end{tikzpicture}
\end{center}

可交换的条件为:

\[
\mathbf{H}(f) \circ (\psi_A \circ \phi_A) = (\psi_B \circ \phi_B) \circ \mathbf{F} (f)
\]

\item{在本节的例子中,我们说在一个偏序集中,如果存在最小值(或最大值),则最小值(或最大值)就是起始对象(或终止对象)。考虑全体偏序集构成的范畴$\pmb{Poset}$,如果存在起始对象,它是什么?如果存在终止对象,它是什么?}

对于$\pmb{Poset}$范畴,对象是偏序集,箭头是单调映射。对于两个偏序集$P, Q$,箭头$P \arrowto{h} Q$使得偏序集$P$中任何两个有序元素$a \leq b$,有$h(a) \leq h(b)$。

这一范畴中的起始对象是空偏序集$0 = \varnothing$。它到任何其它偏序集有唯一的箭头:

\[
\varnothing \longrightarrow P
\]

终止对象是只有一个元素的偏序集$1 = \{\bigstar\}$,偏序关系为$R = \{(\bigstar, \bigstar)\}$,即$\bigstar \leq \bigstar$。任何偏序集$P$到1有唯一的箭头:

\bre
P & \longrightarrow & \{\bigstar\} \\
p & \mapsto & \bigstar
\ere

\item{皮亚诺范畴$\pmb{Pno}$(参见本章第一节的习题2)中,什么样的对象$(A, f, z)$是起始对象?终止对象是什么?}

起始对象是$(\pmb{N}, succ, 0)$,它到任何其它对象唯一箭头是:

\[
(\pmb{N}, succ, 0) \arrowto{\sigma} (A, f, z): \sigma(n) = f^n(z)
\]

终止对象是只有一个元素的对象$1 = (\{\bigstar\}, \bigstar, id)$。任何皮亚诺范畴的对象$(A, f, z)$到终止对象的唯一箭头是:

\[
(A, f, z) \arrowto{\sigma} 1 : \sigma(a) = \bigstar
\]

\item{验证$\pmb{Exp}$的确是一个范畴,指出$id$箭头和箭头的组合。}

我们先看$id$箭头的定义$h \arrowto{id} h$,它使得下面的范畴图可交换:

\begin{center}
\begin{tikzpicture}
  \matrix (m) [matrix of math nodes,
               row sep=3em, column sep=5em, minimum width=2em]{
     A & A \times B & \\
     A & A \times B & C \\};
  \path[-stealth]
    (m-1-1) edge node [left] {$id_A$} (m-2-1)
    (m-1-2) edge node [left] {$id_A \times id_B$} (m-2-2)
    (m-1-2) edge node [above] {$h$} (m-2-3)
    (m-2-2) edge node [below] {$h$} (m-2-3);
\end{tikzpicture}
\end{center}

然后是箭头组合:

$h \arrowto{i} k$和$k \arrowto{j} m$的组合是$j \circ i$使得下面的范畴图交换:

\begin{center}
\begin{tikzpicture}
  \matrix (m) [matrix of math nodes,
               row sep=3em, column sep=5em, minimum width=2em]{
     A & A \times B & \\
     D & D \times B & C \\
     E & E \times B & \\};
  \path[-stealth]
    (m-1-1) edge node [left] {$f$} (m-2-1)
    (m-2-1) edge node [left] {$g$} (m-3-1)
    (m-1-2) edge node [left] {$f \times id_B$} (m-2-2)
    (m-2-2) edge node [left] {$g \times id_B$} (m-3-2)
    (m-1-2) edge node [above] {$h$} (m-2-3)
    (m-2-2) edge node [above] {$k$} (m-2-3)
    (m-3-2) edge node [below] {$m$} (m-2-3);
\end{tikzpicture}
\end{center}

因此对于箭头$h \arrowto{j} k$有:$id_k \circ j = j = j \circ id_h$。并且对于三个箭头有结合律。

\item{反射律$curry\ apply = id$中,$id$的下标是什么?请用另一种方法证明它。}

$id$的下标是二元箭头的类型$A \times B \to C$。

\begin{proof}
\blre
  & curry \circ apply\ f\ a\ b \\
= & curry\ (apply\ f)\ a\ b & \text{组合的定义} \\
= & (apply\ f)\ (a, b) & \text{$curry$的定义}\\
= & f(a, b) & \text{$apply$的定义} \\
= & id_{A \times B \to C}\ f(a, b)
\elre
\end{proof}

\item{我们称下面的等式
\[
(curry\ f) \circ g = curry(f \circ (g \times id))
\]
为柯里化的融合律。请画出它的范畴图并证明它。}

\begin{center}
\begin{tikzpicture}
  \matrix (m) [matrix of math nodes,
               row sep=3em, column sep=5em, minimum width=2em]{
     D & D \times B & \\
     A & A \times B & \\
     C^B & C^B \times B & C \\};
  \path[-stealth]
    (m-1-1) edge node [left] {$g$} (m-2-1)
    (m-2-1) edge node [left] {$curry\ f$} (m-3-1)
    (m-1-2) edge node [left] {$g \times id$} (m-2-2)
    (m-1-2) edge [bend right] node {} (m-3-2)
    (m-1-2) edge node [above] {$f \circ (g \times id)$} (m-3-3)
    (m-2-2) edge node [above] {$f$} (m-3-3)
    (m-3-2) edge node [below] {$apply$} (m-3-3);
\end{tikzpicture}
\end{center}

观察图中$D \times B$,$A \times B$,和$C$这个三角形。我们知道$D \times B \to C$这个组合箭头为$f \circ (g \times id)$。

根据幂对象和转换箭头的定义有:

\[
apply \circ (curry\ f) \circ g = f \circ (g \times id)
\]

由$curry$和$apply$的泛性性质有:

\[
(curry\ f) \circ g = curry (f \circ (g \times id))
\]

\item{画出群的可逆性公理的范畴图。}

可逆性公理表示为:$m \circ (id, i) = m \circ (i, id) = e$

\begin{center}
\begin{tikzpicture}
  \matrix (m) [matrix of math nodes,
               row sep=3em, column sep=5em, minimum width=2em]{
     G & G \times G & G \\
     1 & G & 1 \\};
  \path[-stealth]
    (m-1-1) edge node [above] {$(id, i)$} (m-1-2)
    (m-1-3) edge node [above] {$(i, id)$} (m-1-2)
    (m-1-1) edge node [left] {} (m-2-1)
    (m-1-2) edge node [left] {$m$} (m-2-2)
    (m-1-3) edge node [left] {} (m-2-3)
    (m-2-1) edge node [above] {$e$} (m-2-2)
    (m-2-3) edge node [above] {$e$} (m-2-2);
\end{tikzpicture}
\end{center}

\item{$p$是一个素数,使用群的F-代数,为整数模$p$乘法群(可以参考上一章)定义一个$\alpha$箭头。}

根据群的$\alpha$箭头定义:

\[
\mathbf{F} A \arrowto{\alpha = e + m +i} A
\]
定义整数模$p$乘法群如下:

\bre
e\ () & = & 1 & \text{单位元是1} \\
m(a, b) & = & ab \bmod p & \text{二元运算是模$p$乘法} \\
i(a) & = & a^{p-2} \bmod p & \text{根据费马小定理$a^{p-1} \equiv 1 \mod p$} \\
\ere

\item{参考上一章环的定义,定义环的F-代数。}

环上的代数结构由三部分组成:
\begin{enumerate}[i]
\item 携带对象$R$,用于携带环上代数结构的集合。
\item 多项式函子$\mathbf{F}A = 1 + 1 + A \times A + A \times A + A$。
\item 箭头$\mathbf{F}A \arrowto{\alpha = z + e + p + m + n} A$,它们分别是加法单位元$z$,乘法单位元$e$,加法$p$,乘法$m$,和求反$n$。
\end{enumerate}
这样就定义了一个环上的$F$-代数$(R, \alpha)$。例如当携带对象是整数时,算术运算下的环定义为:

\[\begin{array}{l}
z\ () = 0 \\
e\ () = 1 \\
p(a, b) = a + b \\
m(a, b) = ab \\
n(a) = -a \\
\end{array}\]

\item{F-代数范畴上的$id$箭头是什么?箭头组合是什么?}

从$F$-代数$(A, \alpha)$到其自身的同态映射是$id$箭头。箭头组合是$F$-态射的组合。携带对象间的箭头:$A \arrowto{f} B \arrowto{g} C$使得下面的范畴图可交换:

\begin{center}
\begin{tikzpicture}
  \matrix (m) [matrix of math nodes,
               row sep=3em, column sep=5em, minimum width=2em]{
     \mathbf{F} A & \mathbf{F} B & \mathbf{F} C \\
     A & B & C \\};
  \path[-stealth]
    (m-1-1) edge node [left] {$\alpha$} (m-2-1)
    (m-1-2) edge node [left] {$\beta$} (m-2-2)
    (m-1-3) edge node [left] {$\gamma$} (m-2-3)
    (m-1-1) edge node [above] {$\mathbf{F}(f)$} (m-1-2)
    (m-1-2) edge node [above] {$\mathbf{F}(g)$} (m-1-3)
    (m-2-1) edge node [below] {$f$} (m-2-2)
    (m-2-2) edge node [below] {$g$} (m-2-3);
\end{tikzpicture}
\end{center}

\[
  g \circ f \circ \alpha = \gamma \circ \mathbf{F}(g) \circ \mathbf{F}(f) = \gamma \circ \mathbf{F}(g \circ f)
\]

\item{可否把类自然数函子写成如下递归的形式?谈谈你的看法。
\begin{lstlisting}
data NatF A = ZeroF | SuccF (NatF A)
\end{lstlisting}
}

不可以。考虑携带对象$A$,上述函子$\mathbf{NatF}$是迭代的,它并不将$A$映射为确定的某个对象。事实上,我们希望把它映射为皮亚诺范畴上的对象$(A, f, z)$。

\item{我们可以为$\mathbf{NatF} Int \to Int$定义一个$\alpha$箭头,名叫$eval$:
\[
\begin{array}{l}
eval : \mathbf{NatF} Int \to Int \\
eval\ ZeroF = 0 \\
eval\ (SuccF\ n) = n + 1 \\
\end{array}
\]
如果迭代地将$A' = \mathbf{NatF} A$代入$\mathbf{NatF}$函子$n$次。我们把这样得到的函子记为$\mathbf{NatF}^n A$。试思考能否定义下面的$\alpha$箭头:
\[
\begin{array}{l}
eval : \mathbf{NatF}^n Int \to Int \\
\end{array}
\]
}

\[
\resizebox{\textwidth}{!}{\ensuremath{
\begin{array}{rcll}
eval & : &\mathbf{NatF}^n Int \to Int \\
eval\ ZeroF & = & 0 & \text{$ZeroF$是$\mathbf{NatF}^n Int$的对象}\\
eval\ (SuccF\ ZeroF) & = & 1 & \text{$ZeroF$是$\mathbf{NatF}^{n-1} Int$的对象}\\
eval\ (SuccF\ (SuccF\ ZeroF)) & = & 2 & \text{$ZeroF$是$\mathbf{NatF}^{n-2} Int$的对象}\\
... \\
eval\ (SuccF^{n-1}\ ZeroF) & = & n - 1 & \text{$ZeroF$是$\mathbf{NatF} Int$的对象}\\
eval\ (SuccF^n\ m) & = & m + n \\
\end{array}
}}
\]

\item{对二叉树函子$\mathbf{TreeF}\ A\ B$,固定$A$,利用不动点验证$(\mathbf{Tree}\ A, [nil, branch])$是初始代数。}

令$B' = \mathbf{TreeF}\ A\ B$,然后不断递归地应用到自己,把此结果称为$\mathbf{Fix}\ (\mathbf{TreeF}\ A)$。

\[
\resizebox{\textwidth}{!}{\ensuremath{
\begin{array}{rcll}
\mathbf{Fix}\ (\mathbf{TreeF}\ A) & = & \mathbf{TreeF}\ A\ (\mathbf{Fix}\ (\mathbf{TreeF}\ A)) & \text{不动点的定义} \\
 & = & \mathbf{TreeF}\ A\ (\mathbf{TreeF}\ A (...)) & \text{递归展开} \\
 & = & NilF | BrF\ A\ (\mathbf{TreeF}\ A\ (...))\ (\mathbf{TreeF}\ A\ (...)) & \text{二叉树函子的定义} \\
 & = & NilF | BrF\ A\ (\mathbf{Fix}\ (\mathbf{TreeF}\ A))\ (\mathbf{Fix}\ (\mathbf{TreeF}\ A)) & \text{反向用不动点} \\
\end{array}
}}
\]

比较$Tree A$的定义:

\begin{lstlisting}
data Tree A = Nil | Br A (Tree A) (Tree A)
\end{lstlisting}

故而:$\mathbf{Tree}\ A = \mathbf{Fix}\ (\mathbf{TreeF}\ A)$。初始代数为$(\mathbf{Tree}\ A, [nil, branch])$。

\end{enumerate}

\section{融合}

\begin{enumerate}
\item {验证从左侧叠加也可以表示为$foldr$:
\[
foldl\ f\ z\ xs = foldr\ (b\ g\ a \mapsto g\ (f\ a\ b))\ id\ xs\ z
\]}

为了方便理解,我们将其改写为:

\[\begin{array}{l}
foldl\ f\ z\ xs = foldr\ step\ id\ xs\ z \\
\text{其中:} step\ x\ g\ a = g\ (f\ a\ x)
\end{array}\]

\blre
  & foldl\ f\ z\ [x_1, x_2, ..., x_n] \\
= & (foldr\ step\ id\ [x_1, x_2, ..., x_n])\ z \\
= & (step\ x_1 (step\ x_2 ( ... (step\ x_n\ id))) ...)\ z \\
= & (step\ x_1 (step\ x_2 ( ... (a_n \mapsto id\ (f\ a_n\ x_n)))) ...)\ z \\
= & (step\ x_1 (step\ x_2 ( ...(a_{n-1} \mapsto (a_n \mapsto id\ (f\ a_n\ x_n))\ (f\ a_{n-1}\ x_{n-1}))))...)\ z \\
= & (a_1 \mapsto (a_2 \mapsto ( ... (a_n \mapsto id\ (f\ a_n\ x_n))\ (f\ a_{n-1}\ x_{n-1})) ... (f\ a_2\ x_2)) (f\ a_1\ x_1))\ z \\
= & (a_1 \mapsto (a_2 \mapsto ( ... (a_n \mapsto f\ a_n\ x_n)\ (f\ a_{n-1}\ x_{n-1})) ... )\ (f\ a_1\ x_1))\ z \\
= & (a_1 \mapsto (a_2 \mapsto (... (a_{n-1} \mapsto f\ (f\ a_{n-1}\ x_{n-1})\ x_n)\ ...))\ (f\ a_1\ x_1))\ z \\
= & (a_1 \mapsto f\ (f\ (...(f\ a_1\ x_1)\ x_2)\ ...)\ x_n)\ z \\
= & f\ (f\ (...(f\ z\ x_1)\ x_2)\ ...)\ x_n
\elre

如果把$f$改写成$\oplus$,并写成中缀形式,就能看出$foldl$和$foldr$的区别:

\[
foldl\ \oplus\ f\ z = ((...(z \oplus x_1)\ \oplus x_2)...)\ \oplus x_n
\]

\item {证明以下列表的构建——叠加形式:
\[
\begin{array}{l}
concat\ xss = build\ (f\ z \mapsto foldr\ (xs\ x \mapsto foldr\ f\ x\ xs)\ xss) \\
map\ f\ xs = build\ (\oplus\ z \mapsto foldr\ (y\ ys \mapsto (f\ y) \oplus ys)\ z\ xs) \\
filter\ f\ xs = build\ (\oplus\ z \mapsto foldr\ (x\ xs' \mapsto
  \begin{cases}
     f(x): & x \oplus xs' \\
    \text{否则}: & xs' \\
  \end{cases})\ z\ xs) \\
repeat\ x = build\ (\oplus\ z \mapsto let\ r = x \oplus r\ in\ r) \\
\end{array}
\]
}

首先是多个列表的连接$concat$
\begin{proof}
\blre
  & build\ (f\ z \mapsto foldr\ (xs\ x \mapsto foldr\ f\ x\ xs)\ z\ xss) \\
= & (f\ z \mapsto foldr\ (xs\ x \mapsto foldr\ f\ x\ xs)\ z\ xss)\ (:)\ [] & \text{$build$的定义} \\
= & foldr\ (xs\ x \mapsto foldr\ (:)\ x\ xs)\ []\ xss & \text{$\beta$-规约} \\
= & foldr\ \doubleplus []\ xss & \text{两个列表连接的定义} \\
= & concat\ xss & \text{多列表连接的定义} \\
\elre
\end{proof}

接着是逐一映射$map$
\begin{proof}
\blre
  & build\ (\oplus\ z \mapsto foldr\ (y\ ys \mapsto (f\ y) \oplus ys)\ z\ xs) \\
= & (\oplus\ z \mapsto foldr\ (y\ ys \mapsto (f\ y) \oplus ys)\ z\ xs)\ (:)\ [] & \text{$build$的定义} \\
= & foldr\ (y\ ys \mapsto f(y) : ys)\ []\ xs & \text{$\beta$-规约} \\
= & foldr\ (x\ ys \mapsto f(x) : ys)\ []\ xs & \text{$\alpha$变换,改名字} \\
= & map\ f\ xs & \text{逐一映射的定义} \\
\elre
\end{proof}

接着是过滤操作$filter$

\begin{proof}
\blre
  & build\ (\oplus\ z \mapsto foldr\ (x\ xs' \mapsto
  \begin{cases}
     f(x): & x \oplus xs' \\
    \text{否则}: & xs' \\
  \end{cases})\ z\ xs) \\
= & (\oplus\ z \mapsto foldr\ (x\ xs' \mapsto
  \begin{cases}
     f(x): & x \oplus xs' \\
    \text{否则}: & xs' \\
  \end{cases})\ z\ xs)\ (:)\ [] & \text{$build$的定义} \\
= & foldr\ (x\ xs' \mapsto
  \begin{cases}
     f(x): & x : xs' \\
    \text{否则}: & xs' \\
  \end{cases})\ []\ xs & \text{$\beta$-规约} \\
= & filter\ f\ xs & \text{过滤操作的定义} \\
\elre
\end{proof}

最后是重复操作$repeat$

\begin{proof}
\blre
  & build\ (\oplus\ z \mapsto let\ r = x \oplus r\ in\ r) \\
= & (\oplus\ z \mapsto let\ r = x \oplus r\ in\ r)\ (:)\ [] & \text{$build$的定义} \\
= & (let\ r = x : r\ in\ r) & \text{$\beta$-规约} \\
= & repeat\ x & \text{重复操作的定义} \\
\elre
\end{proof}

\item{化简快速排序算法:
\[
\begin{cases}
qsort\ [] = [] \\
qsort\ (x:xs) = qsort\ [a | a \in xs, a \leq x] \doubleplus [x] \doubleplus qsort\ [a | a \in xs, x < a] \\
\end{cases}\]
}

可以把ZF表达式变换成$filter$。显然这里处理了两遍列表,它们可以合成一次:

\[
\begin{cases}
qsort\ [] & = [] \\
qsort\ (x:xs) & = qsort\ as \doubleplus [x] \doubleplus qsort\ bs \\
\end{cases}
\]

其中:
\[\begin{array}{l}
(as, bs) = foldr\ h\ ([], [])\ xs \\
h\ y\ (as', bs') = \begin{cases}
               y \leq x : & (y:as', bs') \\
               \text{否则}: & (as', y:bs') \\
\end{cases} \\
\end{array}\]

接下来我们可以把两个列表的连接操作化简:

\blre
  & qsort\ as \doubleplus [x] \doubleplus qsort\ bs \\
= & qsort\ as \doubleplus (x : qsort\ bs) \\
= & foldr (:)\ (x : qsort\ bs) (qsort\ as)
\elre

\item{利用范畴论验证融合律的类型限制。提示:考虑向下态射的类型。}

观察范畴图:

\begin{center}
\begin{tikzpicture}
  \matrix (m) [matrix of math nodes,
               row sep=3em, column sep=5em, minimum width=2em]{
     \mathbf{ListF} A\ \lbrack A \rbrack & \lbrack A \rbrack \\
     \mathbf{ListF} A\ B\ & B \\};
  \path[-stealth]
    (m-1-1) edge node [left] {$\mathbf{ListF} A(h)$} (m-2-1)
    (m-1-2) edge node [right] {$h = \lbb \alpha \rbb$} (m-2-2)
    (m-1-1) edge node [above] {$(:) + []$} (m-1-2)
    (m-2-1) edge node [below] {$\alpha = f + z$} (m-2-2);
\end{tikzpicture}
\end{center}

向下态射$\lbb \alpha \rbb$被抽象成从$\alpha$构造某种代数结构$g\ \alpha$。$g$接受一个$F$-代数的$\alpha$箭头,产生结果$B$。$\alpha$箭头是$f : A \to B \to B$和$z : 1 \to B$的和。故其类型为:

\[
g : \forall A. (\forall B. (A \to B \to B) \to B \to B)
\]

构造的定义是$build(g) = g\ (:)\ []$,它把$g$应用到初始代数的$\alpha$箭头上从而构造出初始代数中的对象,也就是列表$[A]$。因而:

\[
build : \forall A. (\forall B. (A \to B \to B) \to B \to B) \to \mathbf{List}\ A
\]

\item{利用融合律化简算式求值的定义$eval = sum \circ map\ (product \circ (map\ dec))$。}

\[
\resizebox{\textwidth}{!}{\ensuremath{
\begin{array}{cll}
  & eval\ es \\
= & sum (map\ (product \circ (map\ dec))\ es) & \text{函数组合} \\
  & \{ \text{$sum$展开为叠加形式,$map$展开为构建形式} \} \\
= & \pmb{foldr}\ (+)\ 0\ (\pmb{build}\ (\oplus\ z\ \mapsto foldr\ (t\ ts \mapsto (f\ t) \oplus ts)\ z\ es)) & \text{令$f = product \circ (map\ dec)$} \\
= & (\oplus\ z \mapsto foldr\ (t\ ts \mapsto (f\ t) \oplus ts)\ z\ es)\ (+)\ 0 & \text{融合律} \\
= & foldr\ (t\ ts \mapsto (f\ t) + ts)\ 0\ es & \text{$\beta$-规约} \\
\end{array}
}}
\]

写成无参数形式就是:

\[
eval = foldr\ (t\ ts \mapsto (f\ t) + ts)\ 0
\]

接下来我们再化简$product \circ (map\ dec)$部分

\[
\resizebox{\textwidth}{!}{\ensuremath{
\begin{array}{cll}
  & (product \circ (map\ dec))\ t \\
= & product\ (map\ dec\ t) & \text{函数组合} \\
  & \{ \text{$product$展开为叠加形式,$map$展开为构建形式} \} \\
= & \pmb{foldr}\ (\times)\ 1\ (\pmb{build}\ (\oplus\ z \mapsto foldr\ (d\ ds \mapsto (dec\ d) \oplus ds)\ z\ t)) \\
= & (\oplus\ z \mapsto foldr (d\ ds \mapsto (dec\ d) \oplus ds)\ z\ t)\ (\times)\ 1 & \text{融合律} \\
= & foldr\ (d\ ds \mapsto (dec\ d) \times ds)\ 1\ t & \text{$\beta$-规约} \\
= & foldr\ ((\times) \circ fork\ (dec, id))\ 1\ t & \text{定义$fork(f, g)\ x = (f\ x, g\ x)$} \\
\end{array}
}}
\]

接着把这个结果代入之前的$f$,得到最终化简的结果:

\[
eval = foldr\ (t\ ts \mapsto (foldr\ ((\times) \circ fork\ (dec, id))\ 1\ t) + ts)\ 0
\]

\item{如何从左侧扩展出所有的算式?}

从左向右扩展时,针对每个数字$d$,有三种选择:

\begin{enumerate}
\item 什么都不插入意味着将数字$d$直接添加到$e_i$的最后一个子算式中的最后一个因子的后面。这样由$f_n \doubleplus [d]$组成一个新的因子。例如$e_i$是算式$1 + 2$,数字$d$是3,将3写在$1 + 2$的后面而什么符号都不插入,这样就得到新算式$1 + 23$;
\item 插入乘号意味着用数字$d$构成一个因子$[d]$,然后将它添加到$e_i$中最后一个子算式的后面。这样由$t_m \doubleplus [[d]]$组成一个新的子算式。具体到$1 + 2$这个例子,我们把3写在它的后面,然后在2和3之间插入一个乘号,这样就得到新算式$1 + 2 \times 3$;
\item 插入加号意味着用数字$d$构成一个子算式$[[d]]$,然后将将它添加到$e_i$的最后面,组成新的算式$e_i \doubleplus [[[d]]]$。具体到$1 + 2$这个例子,我们把3写在它的后面,然后在2和3之间插入一个加号,这样就得到新算式$1 + 2 + 3$。
\end{enumerate}

为了将一个元素加到序列的末尾,我们可以定义一个函数:

\[
append\ x = foldr\ (:)\ [x]
\]

然后我们定义一个函数$onLast(f)$,它把$f$应用到一个序列的最后一个元素上:

\[\begin{array}{l}
onLast\ f = foldr\ h\ [] \\
\text{其中}: \begin{cases}
  h\ x\ [] & = [(f\ x)] \\
  h\ x\ xs & = x : xs \\
\end{cases} \\
\end{array}\]

然后,我们就可以实现这3种情况的扩展:

\lstset{frame = none}
\begin{lstlisting}
add d exp = [((append d) `onLast`) `onLast` exp,
             (append [d]) `onLast` exp,
             (append [[d]]) exp]
\end{lstlisting}

\item{下面定义可以将算式翻译为字符串:
\[
str = (join\ \text{``+''}) \circ (map\ ((join\ \text{``} \times \text{''}) \circ (map\ (show \circ dec))))
\]
其中$show$可以将数字转换为字符串。函数 $join(c, s)$ 将一组字符串$s$用$c$连接起来,例如 $join($``\#''$, [$``abc'', ``def''$]) = $``abc\#def'' 。利用融合律化简$str$的定义。
}

第五章中,我们给出了$join(ws)$的结果,它实际上是用空格分割一组字符串。将空格作为参数,就得到了$join(c, s)$的定义:

\[
join\ c = foldr\ (w\ b \mapsto foldr\ (:)\ (c:b)\ w)\ []
\]

观察$str$的定义,它实际上是两重的$(join\ c) \circ (map\ f)$的形式,即:

\[\begin{array}{l}
str = (join\ c) \circ (map\ f) \\
\text{其中}: f = (join\ d) \circ (map\ g) \\
\end{array}\]

这里$c =$ `+',$d =$ `$\times$',而$g = show \circ dec$。为此我们只要找出$(join\ c) \circ (map\ f)$的简化形式就可以了。

\blre
  & (join\ c) \circ (map\ f)\ es \\
  & \{ \text{$join$展开为叠加,$map$展开为构建形式} \} \\
= & \pmb{foldr}\ (w\ b \mapsto foldr\ (:) (c:b)\ w)\ []\ (\pmb{build}\ (\oplus\ z \mapsto foldr\ (y\ ys \mapsto (f\ y) \oplus ys)\ z\ es)) \\
  & \{ \text{融合律} \} \\
= & (\oplus\ z \mapsto foldr\ (y\ ys \mapsto (f\ y) \oplus ys)\ z\ es))\ (w\ b \mapsto foldr\ (:)\ (c:b)\ w)\ [] \\
  & \{ \text{$\beta$-规约} \} \\
= & foldr\ (y\ ys \mapsto foldr\ (:)\ (c:ys)\ (f\ y))\ []\ es \\
\elre

将之前的加号、乘号,以及$show \circ dec$代入,我们得到结果:

\blre
str & = & foldr\ (x\ xs \mapsto foldr\ (:)\ (`+':xs) ( \\
    &   & \quad foldr (y\ ys \mapsto foldr\ (:)\ (`\times':ys)\ (show \circ dec\ y))\ [])\ [] \\
\elre

\end{enumerate}

\section{无穷}

\begin{enumerate}

\item{第一章中,我们用叠加操作实现了斐波那契数列,如何用$iterate$定义斐波那契数列潜无穷?}

\[
F = (fst \circ unzip)\ (iterate\ ((m, n) \mapsto (n, m + n))\ (1, 1))
\]

例如$take\ 100\ F$

\item{用叠加操作定义$iterate$。}

我们考虑潜无穷流$iterate\ f\ x$,如果将$f$再次应用到每个元素上,并且在最前面添加一个$x$,得到的仍然是这个无穷流。基于这点我们可以定义:

\[
iterate\ f\ x = x : foldr (y\ ys \mapsto (f\ y):ys)\ []\ (iterate\ f\ x)
\]

例如:

\begin{lstlisting}
take 10 $ iter (+1) 0
[0,1,2,3,4,5,6,7,8,9]
\end{lstlisting} %$

\item{利用第四章中介绍的不动点定义,证明$Stream$是$StreamF$的不动点。}

令$A' = \mathbf{StreamF}\ E\ A$,然后不断递归地应用到自己,把此结果称为$\mathbf{Fix}\ (\mathbf{StreamF}\ E)$

\bre
\mathbf{Fix}\ (\mathbf{StreamF}\ E) & = &
    \mathbf{StreamF}\ E\ (\mathbf{Fix}\ (\mathbf{StreamF}\ E)) & \text{不动点的定义} \\
 & = & \mathbf{StreamF}\ E\ (\mathbf{StreamF}\ E\ (...)) & \text{递归展开} \\
 & = & \mathbf{Stream}\ E\ (\mathbf{Stream}\ E\ (...)) & \text{替换名称} \\
 & = & \mathbf{Stream}\ E & \text{反向用$Stream$的定义} \\
\ere

故$Stream$是$StreamF$的不动点。

\item{试定义反折叠$unfold$}

在实现时,通常使用$Maybe$来定义出一个结束条件:
\begin{lstlisting}
unfold :: (b -> Maybe (a, b)) -> (b -> [a])
unfold f b = case f b of
                Just (a, b') -> a : unfold f b'
                Nothing -> []
\end{lstlisting}

\item 数论中的算术基本定理说:任何一个大于1的整数都可以唯一地表示成若干素数的乘积。有一道编程趣题,要求判断一段文字$T$中,是否包含一个字符串$W$的某种排列。试利用算术基本定理,和素数流解决这道题目。

我们的思路是,将每一个不同字符对应到一个素数上去,a对应2,b对应3,c对应5……。这样任意给定一个字符串$W$,不管它是否包含重复的字符,我们都可以把它表示为素数的乘积:

\[
F = \prod p_c , c \in W
\]

我们称其为字符串$W$的数论指纹$F$。如果$W$是空串,我们规定它的指纹等于1。根据整数乘法的交换律,我们知道无论$W$怎样排列,其数论指纹都不变,并且根据算术基本定理,这个数论指纹是唯一的。现在我们就得到了一个特别简洁的解法:我们首先计算出$W$的数论指纹$F$,然后用一个长度为$|W|$的窗口沿着$T$从左向右滑动。一开始我们需要计算$T$在这个窗口内的数论指纹,并和$F$比较,如果相等就说明$T$包含$W$的某种排列。如果不等我们将这个窗口向右滑动一个字符。此时我们可以非常容易地计算新窗口内的数论指纹:只要把滑出的字符对应的素数除掉,再把滑入的字符对应的素数乘上就可以了。任何时候如果新窗口内的数论指纹等于$F$,就说明找到了一个排列。当然为了获得每个不同字符对应的素数,我们还要利用埃拉托斯特尼筛法产生一串素数。下面是一段示例算法:

\begin{algorithmic}
\Function{contains?}{$W, T$}
  \State $P \gets ana \ era \ [2, 3, ...]$ \Comment{素数序列}
  \If{$W = \phi$}
    \State \Return True
  \EndIf
  \If{$|T| < |W|$}
    \State \Return False
  \EndIf
  \State $\displaystyle m \gets \prod P_c, c \in W$
  \State $\displaystyle m' \gets \prod P_c, c \in T[1...|W|]$
  \For{$i \gets |W| + 1$ to $|T|$}
    \If{$m = m'$}
      \State \Return True
    \EndIf
    \State $m' \gets m' \times P_{T_i} / P_{T_{i - |W|}} $
  \EndFor
  \State \Return $m = m'$
\EndFunction
\end{algorithmic}

\item{我们建立了房间和任意旅游团的客人间的一一映射。第$i$号旅游团的第$j$号客人应该入住几号房间?第$k$个房间里住了哪号旅游团的哪位客人?}

按照本章约定,从0开始计数。用数偶$(i, j)$表示第$i$号旅游图的第$j$号客人。我们列出前面的几个客人和房间的对应关系

\btab{c|c|c|c|c|c|c|c|c|c|c|c}
$(i, j)$ & (0, 0) & (0, 1) & (1, 0) & (2, 0) & (1, 1) & (0, 2) & (0, 3) & (1, 2) & (2, 1) & (3, 0) & ... \\
\hline
$k$ & 0 & 1 & 2 & 3 & 4 & 5 & 6 & 7 & 8 & 9 & ... \\
\hline
$i + j$ & 0 & 1 & 1 & 2 & 2 & 2 & 3 & 3 & 3 & 3 & ... \\
\etab

如果同时写下$i+j$的值,我们发现规律是很明显的。共有1个0,2个1,3个2,4个3……这些恰恰是毕达哥拉斯发现的三角形数。记$m = i + j$,对于图中任意格点,它表明在这个点的左下方所有斜线上格点的数目为:$\dfrac{m(m + 1)}{2}$。

在这个点所在的斜线上,如果$m$是奇数则向左上前进,$i$增加、$j$减小;如果是偶数则向右下前进。综合起来,我们得到结果:

\[
k = \dfrac{m(m + 1)}{2} + \begin{cases} m - j: \text{$m$是奇数} \\
j: \text{$m$是偶数} \\
\end{cases}
\]

进一步,我们可以通过$(-1)^m$来简化这个结果:

\[
k = \dfrac{m(m + 2) + (-1)^m (2j - m)}{2}
\]

\item{希尔伯特旅馆第三天的故事的解法并不唯一,根据《无需语言的证明》一书的封面。试根据此图给出另一种编号方案?}

\begin{figure}[htbp]
\centering
\begin{tikzpicture}
  \draw[step=1, very thin, gray] (0, 0) grid (5, 5);
  \draw[->] (-0.25, 0) -- (6, 0) coordinate (x axis);
  \draw[->] (0, -0.25) -- (0, 6) coordinate (y axis);
  \foreach \x in {0, 1, 2, 3, 4, 5}
    \path (\x, -0.25) node[left] {\x};
  \foreach \y in {1, 2, 3, 4, 5}
    \path (-0.25, \y) node[below] {\y};
  \foreach \i / \x / \y in {0/0/0, 1/1/0, 2/1/1, 3/0/1, 4/0/2, 5/1/2, 6/2/2, 7/2/1, 8/2/0, 9/3/0, 10/3/1}{
    \path (\x, \y) coordinate (N\i);
    \fill (N\i) circle (1pt) node[above right=3pt of N\i] {\i};
  }
  \foreach \i in {0,...,9} {
    \pgfmathsetmacro{\j}{\i+1}
    \draw[-latex, thick] (N\i) to (N\j);
  }
\end{tikzpicture}
\caption{对无穷个无穷的另一种编号方案}
\label{fig:NNtoN2}
\end{figure}

如图\ref{fig:NNtoN2}所示,每次沿着折尺形前进,每个折尺上有奇数个点。

\item 令$x = 0.9999....$, 则$10x = 9.9999...$,做减法得$10x - x = 9$,解方程得$x = 1$。因此得到结论$1 = 0.9999...$。这一证明正确么?

正确

\item 在两个镜子中间点燃一支蜡烛,你看到了什么?这是潜无穷还是实无穷?

这支蜡烛在两个镜面间不断反射,产生无穷多的像。也许我们需要考虑光速是有限的,这样它在物理上仍然是潜无穷。

\end{enumerate}

\section{悖论}

\begin{enumerate}
\item {我们可以用语言定义数,例如“最大的两位数”定义了99。定义一个集合,是所有不能用20个以内的字描述的数字。考虑这样一个元素:“不能用20个以内的字描述的最小数”,它是否属于这个集合?}

这是一个罗素悖论,属于或不属于都将导致矛盾。

\item {“这个世界上唯一不变的是变化”——这句话是否是罗素悖论?}

是罗素悖论。

\item {本章开头苏格拉底的话是否是罗素悖论?}

是罗素悖论。

\item{尝试给出费马大定理的印符串。}

我们先要定义出幂运算。

\[\begin{cases}
\forall a: e(a, 0) = S0 & \text{任何数的0次幂为1} \\
\forall a: \forall b: e(a, Sb) = a \cdot e(a, b) & \text{递归} \\
\end{cases}\]

接着就可以定义费马大定理了:

\[
\forall d: \lnot \exists a: \exists b: \exists c: \lnot (d = 0 \lor d = S0 \lor d = SS0) \to e(a, d) + e(b, d) = e(c, d)
\]

\item{尝试用印符推理规则证明加法结合律。}

令人吃惊的是,我们可以证明下面的每一条定理:

\bre
a + b + 0 & = & a + (b + 0) \\
a + b + S0 & = & a + (b + S0) \\
a + b + SS0 & = & a + (b + SS0) \\
... \\
\ere

例如:

\bre
a + b + 0 = a + b = a + (b + 0)
\ere

以及:

\bre
a + b + SS0 & = & SS(a + b + 0) \\
 & = & SS(a + b) \\
 & = & a + SSb \\
 & = & a + (b + SS0) \\
\ere

但是却没有办法证明: $\forall c: a + b + c = a + (b + c)$。

为此必须引入数学归纳法。

\item{利用新加入的归纳规则证明$\forall a: (0 + a) = a$}

首先是0的情况:
\[
0 + 0 = 0
\]

然后假设$(0 + a) = a$成立,我们有:

\bre
(0 + Sa) & = & S(0 + a) & \text{公理3} \\
  & = & Sa & \text{归纳假设} \\
\ere

然后利用归纳规则,有:$\forall a: (0 + a) = a$

\end{enumerate}

\ifx\wholebook\relax \else
\begin{thebibliography}{99}

\bibitem{Lockhart2012}
[美] 保罗$\cdot$洛克哈特 著, 王凌云 译. ``度量——一首献给数学的情歌''. 人民邮电出版社. 2015, ISBN: 9787115393180

\end{thebibliography}

\expandafter\enddocument
%\end{document}

\fi
